\documentclass[12pt]{scrbook}
\usepackage{a4}
\usepackage{geometry}
\geometry{left=25mm, right=40mm, top=25mm, bottom=25mm, footskip=10mm, pdftex} 
\linespread{1.25}

\usepackage{xypic}
\usepackage[latin1]{inputenc}
\usepackage{times}
\usepackage[T1]{fontenc}
\usepackage[english]{babel}
\usepackage{amsmath}
\usepackage{amssymb}
\usepackage{amsfonts}
\usepackage{amsthm}
\usepackage{array, colortbl}
\usepackage{hyperref} 
\usepackage{paralist}
\usepackage{dsfont}
\setlength{\plitemsep}{2pt}

\setlength{\parindent}{3mm}
\setlength{\topmargin}{-2mm}
\setlength{\headheight}{3mm}
\setlength{\headsep}{5mm}
\setlength{\textheight}{232mm}
\setlength{\textwidth}{150mm}
\setlength{\oddsidemargin}{-1mm}
\setlength{\evensidemargin}{-1mm}

\theoremstyle{definition}
\newtheorem{thm}{Theorem}[section]
\newtheorem{defn}[thm]{Definition}
\newtheorem{lemma}[thm]{Lemma}
\newtheorem{cor}[thm]{Corollary}
\newtheorem{rem}[thm]{Remark}
\newtheorem{ex}[thm]{Example}
\newtheorem{remi}[thm]{Reminder}
%\newenvironment{beweis}{\begin{proof}[Beweis]}{\end{proof}}

\newcommand{\quotsim}{/\!\!\sim}
\newcommand{\D}{\ensuremath{\mathbb{D}}}
%\newcommand{\S}{\ensuremath{\mathbb{S}}}
\newcommand{\Q}{\ensuremath{\mathbb{Q}}}
\newcommand{\Z}{\ensuremath{\mathbb{Z}}}
\newcommand{\N}{\ensuremath{\mathbb{N}}}
\newcommand{\R}{\ensuremath{\mathbb{R}}}
\newcommand{\C}{\ensuremath{\mathcal{C}}}
\newcommand{\E}{\ensuremath{\mathcal{E}}}
\newcommand{\B}{\ensuremath{\mathcal{B}}}
\newcommand{\I}{\ensuremath{\lbrack 0,1 \rbrack}}
\newcommand{\lvec}[1]{\ensuremath{\overleftarrow{#1}}}

\author{Marc Lange, 11.12.2010}

\begin{document}
\chapter{Useful fact regarding $BG$}
Let us look at these two maps from $BG$ to $B(G^{op})$:
$$B\iota(\lbrack(g_1,\ldots,g_k), (t_0,\ldots,t_k)\rbrack)= \lbrack(g_1^{-1},\ldots,g_k^{-1}),(t_0,\ldots,t_k)\rbrack$$
and $$\kappa(\lbrack(g_1,\ldots,g_k), (t_0, \ldots, t_k) \rbrack = \lbrack (g_k,\ldots,g_1), (t_k, \ldots, t_0) \rbrack $$.
In \cite{burghelea1985hermitian} we have the repeated claim that these two maps are homotopic and the given homotopy is:
$$K\colon BG \times I \rightarrow B(G^{op})$$
$$(\lbrack(g_1,\ldots,g_k), (t_0,\ldots,t_k) \rbrack,s)$$ $$\mapsto [(g_k, \ldots,g_1, g_1^{-1},\ldots,g_k^{-1}), s(t_k,\ldots,t_0,0_k) + (1-s)(0_k,t_0,\ldots,t_k)]$$
Since it is quite a nuisance to check, if this is well-defined, I do this here for all eternity :D Although in all honesty this is more of a long hint on which indices to calculate to convince yourself that this works.
\section{Well-defined on $BG$-faces $d_i$}
Let $g = (g_1,\ldots,g_k)$ and $t = (t_0,\ldots,t_k)$. Furthermore $0_k$ is a "'fat zero"' with $k$ entries, $\overleftarrow{g}$ as well as $\overleftarrow{t}$ denote the reversing of coordinates and $g^{-1}$ are coordinate-wise inverses. Which makes $K$ a lot more readable as follows: $$K(\lbrack g,t\rbrack ,s)=\lbrack(\overleftarrow{g},g^{-1}),
s(\lvec{t},0_k)+(1-s)(0_k,t)\rbrack$$\newpage 
\subsection{$d_0$}
Let $g = (g_1,\ldots,g_k)$ and $t = (t_0,\ldots,t_{k-1})$.\\
$K(\lbrack g, d_0 t\rbrack,s) = K(\lbrack (g_1,\ldots,g_k) , (0,t_0,\ldots,t_{k-1})\rbrack, s) \\
=\lbrack(\lvec{g},g^{-1}), s(\lvec{t},0,0_k) + (1-s)(0_k,0,t)\rbrack\\
=\lbrack(\lvec{g},g^{-1}), d_k(s(\lvec{t},0_k) + (1-s)(0_k,t) \rbrack\\
=\lbrack d_k(\lvec{g},g^{-1}),s(\lvec{t},0_k) + (1-s)(0_k,t) \rbrack\\
=\lbrack (\lvec{d_0 g},1,{(d_0 g)}^{-1}),s(\lvec{t},0_k) + (1-s)(0_k,t) \rbrack\\
=\lbrack s_{k-1}(\lvec{d_0 g},(d_0 g)^{-1}),s(\lvec{t},0_k) + (1-s)(0_k,t) \rbrack\\
=\lbrack (\lvec{d_0 g},(d_0 g)^{-1}),s_{k-1}(s(\lvec{t},0_k) + (1-s)(0_k,t) ) \rbrack \\
=\lbrack (\lvec{d_0 g},(d_0 g)^{-1}),(s(\lvec{t},0_{k-1})+(1-s)(0_{k-1},t))\rbrack\\
=K(\lbrack d_0g, t \rbrack, s) ~~~~~~\checkmark$
\subsection{$d_k$}
$K(\lbrack d_k g, t \rbrack , s) = \lbrack (\lvec{d_k g }, (d_k g)^{-1}), s(\lvec{t},0_{k-1}) + (1-s) (0_{k-1},t) \rbrack \\
= \lbrack d_0 (\lvec{g},(d_k g)^{-1}),  s(\lvec{t},0_{k-1}) + (1-s) (0_{k-1},t) \rbrack\\
= \lbrack d_0d_{2k}(\lvec{g},g^{-1}),  s(\lvec{t},0_{k-1}) + (1-s) (0_{k-1},t) \rbrack\\
= \lbrack (\lvec{g},g^{-1}), d_{2k}d_0(  s(\lvec{t},0_{k-1}) + (1-s) (0_{k-1},t)) \rbrack\\
= \lbrack (\lvec{g},g^{-1}), d_{2k}(s(0,\lvec{t},0_{k-1}) + (1-s)(0_k,t)) \rbrack \\
= \lbrack (\lvec{g},g^{-1}), s(\lvec{d_k t},0_k) + (1-s) (0_k,d_k t) \rbrack \\
= K(\lbrack g, d_k t \rbrack ,s) ~~~~~~\checkmark$
\subsection{$d_i$ ~~ for $i = 1,\ldots, k-1$}
$K(\lbrack d_i g, t \rbrack, s) = \lbrack (\lvec{d_i g}, (d_i g)^{-1}), s(\lvec{t},0_{k-1}) + (1-s) (0_{k-1},t) \rbrack\\
= \lbrack d_{k-i}(\lvec{g},(d_i g)^{-1}), s(\lvec{t},0_{k-1}) + (1-s) (0_{k-1},t) \rbrack\\
= \lbrack d_{k-i}d_{k+i}(\lvec{g},g^{-1}),s(\lvec{t},0_{k-1}) + (1-s) (0_{k-1},t) \rbrack\\
= \lbrack (\lvec{g},g^{-1}), d_{k+i}d_{k-i}(s(\lvec{t},0_{k-1}) + (1-s)(0_{k-1},t)) \rbrack \\
= \lbrack (\lvec{g},g^{-1}), d_{k+i}(s(\lvec{d_i t},0_{k-1}) + (1-s) ( 0_k,t))\rbrack \\
= \lbrack (\lvec{g},g^{-1}), s(\lvec{d_i t},0_k)  + (1-s)(0_k,d_i t) \rbrack \\
= K(\lbrack g, d_i t \rbrack,s)~~~~~~ \checkmark$ \newpage
\section{Well-defined on $BG$-degeneracies $s_i$}
Now let $g= (g_1, \ldots, g_{k-1})$ and $t = (t_0,\ldots, t_k)$. All the other notations remain.
\subsection{$s_0$}
$K(\lbrack s_0 g, t \rbrack , s) = \lbrack (\lvec{g},1,1,g^{-1}), s(\lvec{t}, 0_k) + (1-s)(0_k,t) \rbrack\\
=\lbrack s_k(\lvec{g},1,g^{-1}),s(\lvec{t}, 0_k) + (1-s)(0_k,t) \rbrack\\
= \lbrack s_k s_{k-1}(\lvec{g},g^{-1}),s(\lvec{t}, 0_k) + (1-s)(0_k,t) \rbrack\\
=\lbrack(\lvec{g},g^{-1}),s_{k-1}s_k(s(\lvec{t}, 0_k) + (1-s)(0_k,t)) \rbrack\\
=\lbrack(\lvec{g},g^{-1}),s_{k-1}(s(\lvec{t},0_{k-1}) +(1-s)(0_k,s_0 t ) )\rbrack \\
=\lbrack(\lvec{g},g^{-1}),(s(\lvec{s_0 t},0_{k-1}) +(1-s)(0_{k-1},s_0 t ) )\rbrack \\ 
=K(\lbrack g, s_0 t \rbrack, s)~~~~~~\checkmark$
\subsection{$s_{k-1}$}
 -- (Attention: this is the right-most $s_i$ according to the choice of $g$ and $t$)\\
$K(\lbrack s_{k-1} g, t \rbrack , s) = \lbrack (1,\lvec{g},g^{-1},1), s(\lvec{t}, 0_k) + (1-s)(0_k,t) \rbrack\\
=\lbrack s_{2k}(1,\lvec{g},g^{-1}),s(\lvec{t}, 0_k) + (1-s)(0_k,t) \rbrack\\
= \lbrack s_{2k}s_0(\lvec{g},g^{-1}),s(\lvec{t}, 0_k) + (1-s)(0_k,t) \rbrack\\
=\lbrack(\lvec{g},g^{-1}),s_0s_{2k}(s(\lvec{t}, 0_k) + (1-s)(0_k,t)) \rbrack\\
=\lbrack(\lvec{g},g^{-1}),s_0(s(\lvec{t},0_{k-1}) +(1-s)(0_k,s_{k-1} t ) )\rbrack \\
=\lbrack(\lvec{g},g^{-1}),(s(\lvec{s_{k-1} t},0_{k-1}) +(1-s)(0_{k-1},s_{k-1} t ) )\rbrack \\ 
=K(\lbrack g, s_{k-1} t \rbrack, s)~~~~~~\checkmark$
\subsection{$s_i$ for $i = 1,\ldots, k-2$}
$K(\lbrack s_i g, t \rbrack,s ) = \lbrack ( \lvec{s_i g}, (s_i g)^{-1}),  s(\lvec{t}, 0_k) + (1-s)(0_k,t) \rbrack\\
= \lbrack s_{k-i-1}(\lvec{g},(s_i g)^{-1}), s(\lvec{t},0_k) + (1-s)(0_k, t) \rbrack \\
= \lbrack s_{k-i-1}s_{k+i-1}(\lvec{g},g^{-1}),s(\lvec{t},0_k) + (1-s)(0_k, t) \rbrack \\
= \lbrack (\lvec{g},g^{-1}),s_{k+i-1}s_{k-i-1}(s(\lvec{t},0_k) + (1-s)(0_k, t))\rbrack \\
= \lbrack (\lvec{g},g^{-1}),s_{k+i-1}(s(\lvec{s_i t},0_k) + (1-s)(0_{k-1}, t))\rbrack \\
= \lbrack (\lvec{g},g^{-1}),s(\lvec{s_i t},0_{k-1}) + (1-s)(0_{k-1},s_i t)\rbrack \\
= K(\lbrack g, s_i t \rbrack,s ) ~~~~~~\checkmark$\newpage
\section{Conclusion}
So $K$ is a well-defined map, and now we can check, what it does\\  $g= (g_1,\ldots,g_k), t= (t_0,\ldots, t_k)$:\\
$K(\lbrack g,t\rbrack,0) = \lbrack(\lvec{g},g^{-1}), (0_k,t) \rbrack = \lbrack(\lvec{g},g^{-1}), {d_0}^k t\rbrack = \lbrack {d_0}^k(\lvec{g},g^{-1}),t\rbrack
=\lbrack g^{-1}, t \rbrack = B\iota \lbrack g,t\rbrack $
and 
$K(\lbrack g,t\rbrack, 1) = \lbrack(\lvec{g},g^{-1}),(\lvec{t},0_k)\rbrack = \ldots = \lbrack \lvec{g}, \lvec{t}\rbrack = \kappa\lbrack g, t \rbrack$
Regarding the "'$\ldots$"': It is mostly clear what happens here, but not quite descriptive if written down. Mainly the $k$ zeroes in the $t$-coordinate allow for cutting off the last $k$ entries in the $g$-coordinate, which is precisely the coordinates of $g^{-1}$.
\bibliographystyle{plain}
\bibliography{bibl.bib} 
\end{document}