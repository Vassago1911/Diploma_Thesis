\chapter{Involutions on Bimonoidal Categories}
So far the involution defined in Richter \cite{richter2010involution} does not directly apply to the preceding chapters.
This section summarises the results and definitions of \cite{richter2010involution} such that later chapters
of this diploma thesis provide non-trivial examples of involutions on $K$-theory, which are induced by involutions of
simplicial rings according to Burghelea and Fiedorowicz \cite{BF} as well as
involutions on bimonoidal categories as defined by Richter \cite{richter2010involution}.

Let $\R$ be a small category in this section. Following \cite[Introduction of Chapter 2]{richter2010involution} and
 \cite[Definition 3.3]{elmendorf2006rings} I define a (strict) bimonoidal category as follows:
\defn{
A strict bimonoidal category $\R$ is a category with two functors $\oplus, \otimes \colon \R\times \R \rightarrow \R$ and two distinguished objects $0,1$
together with a natural transformation $c^{A,B}_\oplus\colon A\oplus B \rightarrow B\oplus A$ and a natural isomorphism
$d_l\colon A\otimes B \oplus A\otimes B' \rightarrow A\otimes (B\oplus B')$, which are subject to the following conditions:
\begin{itemize}
 \item addition and multiplication $\oplus$ and $\otimes$ are strictly associative
 $$A\oplus(B\oplus C)=(A\oplus B)\oplus C ~~~~ A\otimes(B\otimes C)=(A\otimes B) \otimes C,$$
 \item the zero and the unit element are strictly neutral
 $$ A\oplus {0} = {0} \oplus A = A ~~~~ 1\otimes A = A\otimes 1 = A,$$
 \item the zero element strictly multiplies to zero and there is one strict distributivity
 $$A\otimes 0 = 0 \otimes A = 0 ~~~~ A\otimes B \oplus A' \otimes B = (A\oplus A')\otimes B,$$
 \item and the additive twist is a self-inverse map
 $$c^{B,A}_{\oplus}\circ c^{A,B}_{\oplus} = id_{A\oplus B}.$$
\end{itemize}

Furthermore the natural transformations have to satisfy a (long) list of coherence conditions, which are spelled out in Laplaza \cite{laplaza1972coherence} on
pages 31-35 .

\label{bimoncat}
}

The following example illustrates why this is a suitable category analogue for rings.

\ex{To each ring $R$ associate a discrete category $\R_R$ with the following data:
\begin{itemize}
  \item $Ob\R_R = R,$
  \item $r\oplus s := r+s,$
  \item $r\otimes s := rs.$
\end{itemize}\label{ringbiperm}
This is a strict bimonoidal category, where each of the natural transformations is the identity.}

Of course there are also more interesting examples:

\ex{\cite[Example 2.3]{baas2009ring}
For each natural number $n$ denote by $\lbrack n \rbrack$  the following set $\lbrack n\rbrack:=\{1,\ldots,n\}$. Consider the following category $E$ with
\begin{itemize}
\item $Ob E = \{{\lbrack n\rbrack}|n\in\N\}$
\item $E({\lbrack n\rbrack},{\lbrack m\rbrack})=\{f\colon \{1,\ldots,n\}\rightarrow \{1,\ldots,m\}~|~f\in Set({\lbrack n \rbrack},{\lbrack m\rbrack})\},$
\end{itemize}
which is a skeleton of the category of finite sets. Set $\lbrack n \rbrack \oplus \lbrack m \rbrack := \lbrack n+m\rbrack$, which extends to morphisms in the
following fashion (for $f\colon \lbrack n\rbrack \rightarrow \lbrack n'\rbrack$ and $g\colon \lbrack m \rbrack \rightarrow \lbrack m'\rbrack$)
$$(f\oplus g)(i) :=\begin{cases}
                    f(i) & ~~\mathrm{for}~~ 1\leq i \leq n \\
                    n' + g(n-i) & ~~\mathrm{for} ~~ n+1\leq i \leq n+m.
                   \end{cases}$$
Define $c_\oplus\colon \lbrack n+m\rbrack\rightarrow \lbrack m+n\rbrack$ by the formula
$$
\begin{aligned}(c_\oplus)_{n,m} \colon \lbrack &n+m\rbrack \rightarrow \lbrack m+n\rbrack\\
 &i \mapsto {\begin{cases}
			 m+i ~~~ &  \mathrm{~~for~~} 1\leq i\leq n\\
                         i-n ~~~ &\mathrm{~~for~~} n+1\leq i \leq n+m,
                       \end{cases}}
\end{aligned}
$$
which is obviously natural in $n$ and $m$. Furthermore observe that $\oplus$ is strictly associative and $\lbrack 0 \rbrack := \emptyset$ is a strict unit with respect to $\oplus$.

Consider the multiplication $\lbrack n \rbrack \otimes \lbrack m \rbrack := \lbrack nm\rbrack$. This is extended to morphisms by choosing a natural bijection $\lbrack n \rbrack \times \lbrack m \rbrack \cong \lbrack nm\rbrack$ and applying morphisms componentwise,
which I will not display in detail. The multiplication has a twist $c_\otimes$ as well, is strictly associative and has $\lbrack 1\rbrack$ as a strict unit.

Restricted to isomorphisms this gives $$E(\lbrack n\rbrack,\lbrack m\rbrack)=\begin{cases}
                                                                                                       \Sigma_n &\mathrm{~~for~~} m=n\\
												       \emptyset & \mathrm{~~otherwise~~}
                                                                                                      \end{cases}$$}

\ex{Depending on the context the objects $\lbrack n\rbrack$ can also be understood as dimensions or ranks of modules to give another category. The particular example $\mathcal{V}$ of Baas, Dundas and Rognes \cite{baas2004two} is given by
$$
\mathcal{V}(\lbrack n \rbrack, \lbrack m \rbrack) :=\begin{cases}
                                                     U(n) & \mathrm{~for~} n=m\\
						     \emptyset & \mathrm{~otherwise},
                                                    \end{cases}
$$
which is a skeleton of the category finite dimensional complex (unitary) vector spaces with just isomorphisms, which preserve the scalar product.\\
}

There is a category of matrices for strict bimonoidal categories:

\defn{The category of $n\times n$-matrices over $\R$, denoted as $M_n(\R)$, is the following
\begin{itemize}
 \item Objects are matrices of objects in $\R$ $$Ob M_n(\R) := \{(A_{i,j})_{i,j=1,\ldots,n}| A_{i,j}\in Ob\R\},$$
 \item Morphisms are matrices of morphisms between the respective components
$$Mor(A,B) := \{(\varphi_{i,j})_{i,j=1,\ldots,n}| \varphi_{i,j} \in \R(A_{i,j},B_{i,j})\}.$$
\end{itemize}
}

Checking the following lemma from \cite{richter2010involution} is tedious, but straightforward:

\lemma{\cite[Lemma 2.2]{richter2010involution}
For a bimonoidal category $(\R,\oplus,0,c_\oplus,\otimes,1_\R)$ the category of $n\times n$-matrices is a monoidal category with the usual
matrix multiplication
$$(A\cdot B)_{i,j}:=\bigoplus\limits_{k=1}^n A_{i,k}\otimes B_{k,j}.$$
Its unit is the unit matrix $1_n$ with $1_\R$ on the diagonal and zeroes everywhere else. Its associator $\alpha$ can be given by the distributivity morphisms in $\R$,
whereas the left and right unitor morphisms $\lambda\colon 1_n \cdot A \rightarrow A$ and $\rho\colon A\cdot 1_n \rightarrow A$ are identities.\hfill$\Box$
}

\rem{
The fact that $\alpha$ can be expressed by distributivity morphisms on the components, in particular implies that in the case, where both distributivity transformations are identities, the monoidal category
of matrices $M_n(\R)$ will be strict w.r.t. $\otimes$ as well.
}

Let $\R$ be a small bimonoidal category. Then its set of path components $\pi_0\R = \pi_0(|B\R|)$ is a ring except for additive inverses.
(Also called ring without negatives, rig.) Since the sum $\oplus$ induces the structure of an abelian monoid, the usual group completion with respect to 
$\oplus$, i.e. $Gr(\pi_0(\R))= (-\pi_0\R)\pi_0\R$, yields a ring. Set $R:=(-\pi_0 \R)\pi_0 R$ and call $R$ the ring associated to the bimonoidal category $\R$.

Denoting by $GL_n(\R)$ the matrices which have inverse matrices, would yield far too few matrices to be interesting for a general bimonoidal
category. For example, for $\N$ regarded as a bimonoidal category this would only yield permutation matrices. In order to get some more variety,
weaken the invertibility as follows:

\defn{
The monoid of weakly invertible $n\times n$-matrices over the rig $\pi_0(\R)$, denoted by $GL_n(\pi_0(\R))$ is defined to be matrices in $M_n(\pi_0(\R))$,
which are invertible if included into matrices over the ring associated to $\R$, i.e. in $M_n(R)=M_n(Gr(\pi_0\R))$.
}

With this define the category of weakly invertible matrices:

\defn{
The category of weakly invertible $n\times n$-matrices over $\R$ is defined as the full subcategory of $M_n(\R)$ with matrices $A$ such that the projection
to $\pi_0$-classes $\lbrack A \rbrack$ is in the weakly invertible matrices over $\pi_0(\R)$, i.e. $GL_n(\pi_0(\R))$. Denote this category by $GL_n(\R)$.
}

Less formal, since $\pi_0(\R)$ is usually just a rig and not a ring, this additional step just collects all matrices which are invertible
up to connecting chains of morphisms.

\lemma{\cite[Remark between Definition 2.4 and Definition 2.5]{richter2010involution}
Matrix multiplication respects weak invertibility, that is $\cdot\colon M_n(\R)\times M_n(\R) \rightarrow M_n(\R)$ restricts to $GL_n(\R)$.
\begin{proof}
First note that, if $\R$ is small, then each of the defined matrix categories is small as well, hence intersections are defined.
The preceding sequence of definitions then gives the equivalences
$$\begin{aligned}
A\in GL_n(\R)&\Leftrightarrow A\in M_n(\R) \wedge \lbrack A \rbrack \in GL_n(\pi_0(\R))
\\ &\Leftrightarrow A\in M_n(\R) \wedge \lbrack A \rbrack \in M_n(\pi_0(\R)) \wedge \lbrack A \rbrack \in GL_n((-\pi_0(\R)\pi_0(\R)))\\
 &\Leftrightarrow A\in M_n(\R) \wedge \lbrack A \rbrack \in GL_n((-\pi_0(\R)\pi_0(\R)))
  \end{aligned}
$$
and both of these conditions are compatible with matrix multiplication.
\end{proof}
}

\rem{
The stabilisation of $GL_n(R)$ for a ring $R$ (cf. Definition \ref{stableGL}) generalises to this case by mimicking
the stabilisation of linear groups as follows. Define on objects
$$
\begin{aligned}J_n\colon &Ob(GL_n(\R))\rightarrow Ob(GL_{n+1}(\R))\\
&J_n(A):={\begin{pmatrix}
         A & 0_n \\ 0_n^t & 1_\R
        \end{pmatrix}}
\end{aligned}$$
(where $0_n^t$ denotes the horizontal zero $n$-tuple) and extend this on morphisms in the obvious fashion via
$$
J_n(f_{i,j}\colon A\rightarrow B)_{k,l}:= \begin{cases}
                                           f_{k,l}\colon A_{k,l} \rightarrow B_{k,l}  & ~~~ \mathrm{for~~} 0\leq k,l \leq n \\
					   id_{1_\R} &~~~ \mathrm{for~~} k=l=n+1\\
					   id_{0_\R} &~~~ \mathrm{otherwise}
                                          \end{cases}
$$
This defines a functor and hence defines in the usual fashion $GL(\R)$ as the colimit in small categories
$GL(\R):=\mathrm{colim_\mathcal{N}}GL_n(\R)$, which still is a monoidal category.
}

\section{Bar Construction for Monoidal Categories}
From \cite{baas2004two} I take the following bar construction for monoidal categories (no strictness assumed), so in this context for $GL(\R)$.
\defn{\cite[Definition 2.5]{richter2010involution} Let $(\C,\cdot,1_\C,\alpha,\lambda,\rho)$ be a monoidal category (for $\alpha$ the associativity transformation, $\lambda, \rho$ the unit transformations).
Let $B_q(\C)$ be the following category:
\begin{itemize}
 \item Its objects are given by triangular matrices of objects in $\C$, such that for each $0\leq i<j\leq q$ there is an object $A^{i,j}\in \C$, i.e.
$$
\begin{pmatrix}
 A^{0,1} & A^{0,2} & \ldots & A^{0,q}\\
         & A^{1,2} & \ldots & A^{1,q}\\
         &         & \ddots & \vdots   \\
         &         &        & A^{q-1,q}
\end{pmatrix}.
$$
 \item Furthermore for each $0\leq i < j < k \leq q$ there is a (chosen) isomorphism
$$\varphi^{i,j,k}\colon A^{i,j}\cdot A^{j,k}\rightarrow A^{i,k}$$
subject to the coherence
$$\xymatrix{
(A^{i,j}A^{j,k})A^{k,l}\ar[rr]^{\alpha}\ar[d]^{\varphi^{i,j,k}\cdot id}&& A^{i,j}(A^{j,k}A^{k,l})\ar[d]^{id\cdot \varphi^{j,k,l}}\\
A^{i,k}A^{k,l}\ar[r]^{\varphi^{i,k,l}} & A^{i,l} & A^{i,j}A^{j,l}.\ar[l]_{\varphi^{i,j,l}}
}$$
\item A morphism $f$ in $B_q(\C)$ is a set of morphisms $f^{i,j}\colon A^{i,j}\rightarrow B^{i,j}$ for each pair $(i,j)$ satisfying $0\leq i < j \leq q$, such that for all $0\leq i < j < k \leq q$ and $\psi_{i,j,k}$ the isomorphisms of $B$ the maps $f^{i,j}$ satisfy the following coherence
$$f^{i,k}\varphi^{i,j,k}=\psi^{i,j,k}(f^{i,j}\cdot f^{j,k}).$$
\end{itemize}
}

\ex{
The isomorphisms $\varphi^{i,j,k}$ are a necessary part of the data because of the additional choices which are
involved by a non-strict associativity. The $1$-simplices are the objects
of $\C$, so no additional data. In $B_2\C$ there are typical simplices
$$
\begin{pmatrix}
a & ab\\
  & b
\end{pmatrix},
$$
which are one specific instance of a chosen representative given the diagonal entries $a,b$.
But for objects in $B_3\C$ and even fixed diagonal objects $a,b,c$ and binary products $ab, bc$, this triangle can only be built up as follows
$$\begin{pmatrix}
a & ab & ?\\
  & b  & bc\\
  &    & c
\end{pmatrix},
$$
which is the first occasion, where in the top right there is a choice of a representative for the triple product involved. The first coming to mind might
be $a(bc)$ and $(ab)c$ and both are isomorphic via $\alpha$, but there might be even more and the $\varphi^{i,j,k}$ fix the chosen isomorphisms on the way.\\
}

I will not prove the next lemma, but I want to exhibit the statement very clearly.

\lemma{
The categories $B_q(\C)$ form a simplicial category with the following face and degeneracy functors
$$
\begin{aligned}
 d_i\colon& B_q(\C)\rightarrow B_{q-1}(\C) ~~ i=0,\ldots,q\\
 &(d_i(A_{k,l}))_{m,n}=A_{\delta_i(m),\delta_i(n)}\\
 s_i\colon& B_q(\C)\rightarrow B_{q+1}(\C) ~~~i=0,\ldots,q\\
 &(s_i(A_{k,l}))_{m,n}=A_{\sigma_i(m),\sigma_i(n)},
\end{aligned}
$$
where $\delta_i\colon \lbrack q-1 \rbrack \rightarrow \lbrack q \rbrack$ is the monotonic map that skips $i$ and $\sigma_i\colon \lbrack q+1 \rbrack \rightarrow \lbrack q \rbrack$
is the monotonic map that hits $i$ (and only $i$) twice. Since degeneracies $s_i$ might be ``hitting twice'', I use the convention $A_{i,i}=1_\C$ for each $A$ and $i$.

The isomorphisms for $d_i(A)$ are just the isomorphisms $\varphi^{\delta_i(j),\delta_i(k),\delta_i(l)}$, since $d_i$ does not change anything about strictness of $0\leq j<k<l\leq q$.
By the convention to take $A^{i,i}=1_\C$ either $\lambda\colon 1\cdot c \rightarrow c$ or $\rho \colon c \cdot 1 \rightarrow c$ are natural choices for isomorphisms $\varphi$,
according to the position of the unit in the degenerate simplex. This is coherent by the coherence of the left and right unit transformation given by $\C$ being a monoidal category.

The extension of $d_i$ and $s_i$ to morphisms is just restricting a given $(f^{i,j})$ or inserting $f^{i,i}=id_{1_\C}$ at equal indices.
\hfill $\Box$
}

For strict monoidal categories however one can formally mimic the bar construction given in definition \ref{topbarg} and get the following result:

\thm{\cite[Prop 3.9]{baas2004two}\label{barstrict}
The bar construction $B\C$ is equivalent to the strict bar construction $\lbrack n \rbrack \mapsto \C^n_s$ for any strictly
monoidal rigidification $\C_s$ of $\C$.\phantom{blubbel}\hfill$\Box$
}

\rem{One could either directly use the bar construction given for monoidal categories
or strictify the monoidal category first, i.e. replace the associativity, left and right unit transformation by identities
in an organised fashion, and then apply the usual bar construction (definition \ref{topbarg}). This theorem implies that both approaches yield equivalent results.}

\rem{
In order to avoid confusion let me point out explicitly that there is a functor $U\colon \mathrm{CAT} \rightarrow Sets$ from the (1-)category of small categories to sets,
which just sends each category to its set of objects and just forgets morphisms. This then extends to a forgetful functor $U\colon s\mathrm{CAT} \rightarrow sSets$ from
simplicial categories to simplicial sets, which defines the geometrical realisation of $B_\bullet \C$ by setting $|B\C| := |UB\C|$.
}

\section{$K$-Theory of a Strict Bimonoidal Category}
By the preceding section there already is a classifying space associated to the category of weakly invertible matrices over a bimonoidal category, but so
far there is no obvious extension to the meaning of using a Plus-construction $|BGL(\R)|^+$. Fortunately the additional simplices do not affect the
first homotopy group. The following lemma from Baas, Dundas and Rognes \cite{baas2004two} simplifies the proof of lemma \ref{nixneues}.

\lemma{\cite[Proposition 5.3]{baas2004two} Let $B$ be a rig with additive Grothendieck group completion $A := Gr(B)$. Then the rig-homomorphism $B \rightarrow A$ induces a
weak equivalence of classifying spaces $$|BGL(B)| \rightarrow |BGL(A)|,$$ where $GL(B)$ denotes weakly invertible matrices over a rig.
\phantom{blubbel}\hfill$\Box$}

The fundamental group of $BGL(\R)$ can hence be described explicitly by its associated ring $R$:

\lemma{\label{nixneues} Let $\R$ be a bimonoidal category and $R:=(-\pi_0(\R))\pi_0(\R)$ its associated ring.

(1) The natural projection
$$ GL_n(\R) \rightarrow GL_n(R)$$
from weakly invertible $\R$-matrices to the general linear group over $R$ induces a natural simplicial map of the (underlying set of the) bar construction
of the monoidal category $GL_n(\R)$ to the ordinary bar construction of the group $GL_n(R)$
$$p\colon B_\bullet(GL_n(\R)) \longrightarrow B_\bullet(GL_n(R))$$
given by projecting to diagonal entries
$$
p\left(\begin{pmatrix}
       A^{0,1} & A^{0,2} & \ldots & A^{0,q}\\
               & A^{1,2} & \ldots & A^{1,q}\\
               &         & \ddots & \vdots \\
               &         &        & A^{q-1,q}
       \end{pmatrix},(\varphi^{i,j,k})\right) := (\lbrack A^{0,1}\rbrack, \lbrack A^{1,2} \rbrack, \ldots, \lbrack A^{q-1,q} \rbrack)
$$

(2) Stabilisation of $GL_n(\R)$ and $GL_n(R)$ with regard to $n$ and geometrical realisation yield an isomorphism of fundamental groups
$$p_* \colon \pi_1(|BGL(\R)|) \rightarrow \pi_1(|BGL(R)|) = GL(R).$$ In particular there is a natural inclusion of
the elementary matrices over $R$ as a perfect normal subgroup of $\pi_1|BGL(\R)|$, i.e. $E(R) \hookrightarrow \pi_1(|BGL(\R)|)$.

\begin{proof}
(1) Since it is not an ordinary induced map between bar constructions of the same kind, I explicitly present the compatibility with face maps in
the case $l=1,\ldots,q-1$. Note that for matrices in $B_q(GL_n(\R))$ the associated isomorphisms give the
relation $\lbrack A^{i,j}\rbrack \lbrack A^{j,k} \rbrack = \lbrack A^{i,k} \rbrack$ in $\pi_0(\R)$ for indices $i<j<k$, which implies
$$
\begin{aligned}
 p(((d_l(A))^{i,j})_{0\leq i<j\leq q-1}) &= p(((A)^{d_l(i),d_l(j)})_{0\leq i<j \leq q-1})\\
&=(\lbrack A^{0,1}\rbrack,\ldots, \lbrack A^{l-1,l+1} \rbrack, \ldots,\lbrack A^{q-1,q}\rbrack)\\
&=(\lbrack A^{0,1}\rbrack,\ldots, \lbrack A^{l-1,l} \rbrack \lbrack A^{l,l+1} \rbrack, \ldots, \lbrack A^{q-1,q} \rbrack )\\
&=d_l(\lbrack A^{0,1}\rbrack,\ldots, \lbrack A^{q-1,q}\rbrack)\\
&=(d_l\circ p)(A).
\end{aligned}
$$
The other cases are also readily derived from the existence of those morphisms.

(2) By the preceding lemma inspect
$$p\colon |B_\bullet(GL(\R))| \longrightarrow |B_\bullet(GL(\pi_0(\R)))|$$
Without loss of generality assume that a given loop $\bar\gamma \colon \S^{1} \rightarrow |BGL(\pi_0(\R))|$ is a cellular map,
such that $\bar\gamma$ passes through finitely many $1$-cells, since $\S^1$ is compact. More precisely, denote
the unique $0$-cell of $|BGL(\pi_0(\R))|$ by $*$, then the image of the loop except for $*$, i.e. $\bar\gamma(\S^1)\setminus *$,
can be decomposed into a sequence of finitely many $1$-cells. In particular for $\theta \in \S^1$ define
$\lbrack A^{\bar\gamma(\theta)}\rbrack$ as either the unique cell of $\bar\gamma(\theta)$ or as $*$.
It is clear that between two occurrences of the base-point $\lbrack A^{\bar\gamma(\theta)}\rbrack$ is constant.
Choose a sequence of representatives $(A^{\gamma(\theta)})$ with $A^{\gamma(\theta)}=*\in BGL(\R)$, if $\bar\gamma(\theta)=* \in |BGL(\pi_0(\R))|$
and $p(A^{\gamma(\theta)}) = \lbrack A^{\gamma(\theta)}\rbrack$, which are chosen to be locally constant as well.
This gives a map
$$\gamma\colon \S^1 \rightarrow |BGL(\R)| $$
by setting $\gamma(\theta):=\lbrack A^{\gamma(\theta)} , (\bar\gamma)_2(\theta)\rbrack,$
where $(\bar\gamma)_2$ denotes the simplex coordinate in the image of $\Delta^1$ modulo identification.
This is a well-defined and hence a continuous map $\gamma\colon \S^1 \rightarrow |BGL(\R)|,$ which satisfies $p\circ \gamma = \bar\gamma$ per construction.
Thus the induced map $p_* \colon \pi_1(|BGL(\R)|) \rightarrow \pi_1(|BGL(\pi_0(R))|)$ is surjective.

By lifting homotopies in the same fashion with the useful choice
$$p\begin{pmatrix}
A_1 & A_1A_2\\ & A_2
\end{pmatrix}=(\lbrack A_1 \rbrack, \lbrack A_2 \rbrack),$$
the induced map of the projection is injective on $\pi_1$ as well. Therefore on fundamental groups there is an isomorphism
$$p_*\colon \pi_1(|BGL(\R)|) \rightarrow \pi_1(|BGL(R)|) = GL(R).$$
\end{proof}}

Another strategy to prove the preceding lemma is to apply the result \ref{barstrict} and try to sufficiently understand the strictified monoidal category $GL(\R)_s$. Since I only need
this result in degree $1$, lemma \ref{nixneues} allows to associate a $K$-theory space to a bimonoidal category:

\defn{
Let $\R$ be a bimonoidal category and $R=(-\pi_0(\R))\pi_0(\R)$ its associated ring. By the preceding lemmas there is a natural inclusion
$E(R)\rightarrow \pi_1|BGL(\R)|$ and hence the plus-construction can be used with respect to the perfect normal subgroup $E(R)$ to define
$$K(\R)=|BGL(\R)|^+$$
and thus the $K$-groups by
$$K_n(\R):=\pi_n(K(\R)) ~~~ (n\geq 1).$$
}

\rem{
Again recall that I do not define $K_0$ here, since $K(\R)$ is evidently connected.
}

\section{Bimonoidal Categories with Involution}
I will not go through all the details of the construction of an induced involution for a category with involution, since it is very explicitly given in Birgit Richter's paper
\cite[Chapter 4]{richter2010involution}. But I will state the relevant definitions to give the comparison theorem from \cite[Proposition 4.12]{richter2010involution}.

\defn{
An anti-involution in a strict bimonoidal category $\R$ consists of a functor $\zeta\colon \R \rightarrow \R$ with $\zeta\circ\zeta= id$ together with
natural isomorphisms $$\mu\colon \zeta(A\otimes B) \rightarrow \zeta(B)\otimes \zeta(A)$$ for all $A,B\in \R$. These have to satisfy the following conditions
The functor $\zeta$ is strictly symmetric monoidal with respect to $(\R,\oplus, O_\R, c_\oplus)$, fixes the unit by $\zeta(1_\R)=1_\R$ and
 $\mu_{1_\R,A}=id_A=\mu_{A,1_{\R}}$ and satisfies coherences spelled out in \cite[Definition 3.1]{richter2010involution}. Denote a
bimonoidal category with involution by $(\R,\zeta,\mu)$.}

\rem{Recall that the definition \ref{bimoncat} of strict bimonoidal categories requires one of the distributivity morphisms to be strict while the other distributivity is not.
This hinders a generalisation of the interpretation of anti-homomorphisms as morphisms into opposite structures. Opposing
the multiplication to a category $\R^{op_{\otimes}}$ does not give a strict bimonoidal category in the sense of definition \ref{bimoncat}.
Relaxing each distributivity to be just isomorphisms helps in this specific spot, but I will not elaborate on this.}

I define the induced involution with a certain ignorance to all the technicalities involved in \cite{richter2010involution}.

\defn{\label{indinvbimon}
For $\R$ a bimonoidal category with involution $\zeta\colon \R \rightarrow \R$ let
$$\zeta_*\colon |BGL(\R)| \rightarrow |BGL(\R)|$$
be given by
$$
\zeta_* \left[ {\begin{pmatrix}
                         A^{0,1} &\ldots & A^{0,q}\\
                                 &\ddots & \vdots \\
                                 &       & A^{q-1,q}
               \end{pmatrix}},                          (t_0,\ldots,t_q) \right]$$
$$~~~~~~~~~~~~~~~~~  :=\left[  \begin{pmatrix}
														        	         (A^{q-1,q})^t &\ldots & \zeta(A^{0,q})^t\\
														            	                   &\ddots & \vdots \\
														                	               &       & \zeta(A^{0,1})^t
												                   	     \end{pmatrix} , (t_q,\ldots,t_0)  \right]
$$
It is quite clear that the natural inclusion of elementary matrices over $R$
is respected by this involution, so there is an induced map on $K$-spaces
$$\zeta_*\colon K(\R)=|BGL(\R)|^+ \rightarrow |BGL(\R)|^+.$$
Call this the involution on $K$-theory induced by $\zeta\colon \R \rightarrow \R$.
}

\rem{
This means the involution is induced by transposing each matrix (beware that transposition is a functor now), applying the involution to each matrix
componentwise and reversing the triangle (of matrices) along the secondary diagonal.
Of course the involved isomorphisms are to be changed accordingly as well.
}

In order to compare the involution on the linear group of a ring with involution with the one on a bimonoidal category with involution,
one obviously needs to replace the inverse map. This is provided by the following lemma from \cite{BF} (in the proof of
Proposition 4.5), which compares the canonical homeomorphism $|BG| \rightarrow |B(G^{op})|$ for a group $G$ given by the inverse map
to another one, which generalises to monoids easily.

\lemma{\cite[Proof of Proposition 4.5]{BF} There is a homotopy between the homeomorphism\label{homotopyinvrev}
$$\kappa \colon  |BG| \rightarrow |B(G^{op})|$$ given by $\kappa(\lbrack (x_1,\ldots,x_n),(t_0,\ldots,t_n) \rbrack) := \lbrack (x_n,\ldots,x_1),(t_n,\ldots,t_0) \rbrack$
and the canonical homeomorphism $|B\iota|\colon |BG| \rightarrow |BG^{op}|$ given by the induced map of the inverse map.

\begin{proof}(Sketch)
The homotopy $H\colon |BG|\times I \rightarrow |BG^{op}|$ is given by
$$\begin{aligned}
  H(\lbrack &(g_1,\ldots,g_n),(t_0,\ldots,t_n) \rbrack,s) \\&:= \lbrack (g_n,\ldots,g_1,{g_1}^{-1},\ldots,{g_n}^{-1}),
  s(t_n,\ldots,t_0,0_n) + (1-s)(0_n,t_0,\ldots,t_n) \rbrack
  \end{aligned}
$$
where $0_n$ denotes a zero tuple with $n$ positions. It is tedious to check but true that this is well-defined and a homotopy between the given maps.
\end{proof}
}

The following results compare the induced involutions on $K$-theory of strict bimonoidal categories and the $K$-theory of rings.
Of course the comparison should be induced by the natural projection $p\colon GL_n(\R)\rightarrow GL_n(R)$. The induced involution defined in \ref{indinvbimon} extends
the earlier definition for rings given in \ref{indinvK}, if the projection induces a map on $K$-groups respecting the involution.

\thm{\cite[Corollary 4.11]{richter2010involution} \label{projecting}
For $(\R,\zeta,\mu)$ a bimonoidal category with involution let $R=(-\pi_0(\R))\pi_0(\R)$ be its associated ring. Then $R$ is a ring with anti-involution
and the induced map of projecting to components in $\pi_0$  $$p\colon |BGL(\R)| \rightarrow |BGL(R)|$$ induces a map of
$K$-groups $$K_i(\R)\rightarrow K_i(R)$$
respecting the two given induced involutions. More precisely, the diagram $$\xymatrix{
|BGL(\R)|\ar[r]^p \ar[d]^{\zeta_*}&|BGL(R)|\ar[d]^{\zeta_*}\\
|BGL(\R)|\ar[r]^p&|BGL(R)|\\
}$$
commutes.

\begin{proof}
Study the following relations
$$\begin{aligned}&(p \zeta_*)\left\lbrack \begin{pmatrix}
                       A^{0,1} &\ldots & A^{0,q}\\
                               &\ddots & \vdots \\
                               &       & A^{q-1,q}
                      \end{pmatrix},(t_0,\ldots,t_q)\right\rbrack\\
=&p\left\lbrack \begin{pmatrix}
                       \zeta(A^{q-1,q})^t &\ldots & \zeta(A^{0,q})^t\\
                               &\ddots & \vdots \\
                               &       & \zeta(A^{0,1})^t
                      \end{pmatrix},(t_q,\ldots,t_0)\right\rbrack\\
=&\lbrack (\lbrack \zeta(A^{q-1,q})^t\rbrack, \ldots, \lbrack \zeta(A^{0,1})^t \rbrack),(t_q,\ldots, t_0)\rbrack\\
\end{aligned}$$
But $R$ is an ordinary ring, so the fact $\kappa \simeq |B\iota|$ by \ref{homotopyinvrev} gives the following equality on homotopy groups
$$
\begin{aligned}
p_* \circ \zeta_* &= \kappa_* \circ |BT|_* \circ |B\zeta|_* \\&= |B\iota| \circ |BT|_* \circ |B\zeta|_*\\ &= |B(\iota\circ T \circ \zeta)|_*,
\end{aligned}$$
which is the induced involution defined for a ring with involution.
\end{proof}
}

In particular, the associated bimonoidal category to a ring with involution yields the same $K$-theory.

\cor{\cite[Proposition 4.12]{richter2010involution}
Let $R$ be any ring with involution and $\R_R$ its associated bimonoidal category (cf. Example \ref{ringbiperm}), then there is an isomorphism of
$K$-groups compatible with the induced involutions. The isomorphism is induced by the projection $p\colon Ob(\R_R) \rightarrow R$ and gives
$$K_*(\R_R)\cong K_*(R).$$
\begin{proof}
Inspect the bar construction of matrices over $\R_R$ first. For the bar construction over the discrete category $\R_R$ associated to the ring $R$
the existence of a morphism $\varphi^{i,j,k}\colon A^{i,j}A^{j,k}\rightarrow A^{i,k}$ is equivalent to the
statement $A^{i,j}A^{j,k} = A^{i,k}$. So the isomorphisms are redundant information, thus each simplex has the following form
$$
\begin{aligned}
\begin{pmatrix}
  A^{0,1} & A^{0,1}A^{1,2} & \ldots & A^{0,1}A^{1,2}\ldots A^{q-1,q}\\
          & A^{1,2} &\ldots  & A^{1,2}\ldots A^{q-1,q}\\
          &         &\ddots  & \vdots\\
          &         &        & A^{q-1,q}
\end{pmatrix}.
\end{aligned}
$$
This implies that each $q$-simplex is uniquely determined by its diagonal. Furthermore the equality $R = \pi_0(\R_R) = Gr(\pi_0(\R_R))$
implies that restricting the matrix components via $p$ is just a plain identity.

This extends to an isomorphism of simplicial sets $Ob(B_\bullet(GL(\R_R))) \cong B_\bullet(GL(R))$ and theorem \ref{projecting}
implies that restricting the matrix entries induces a morphism of $K$-groups, which is compatible with the induced involutions,
thus the claim follows.
\end{proof}
}

Theorem \ref{projecting} of course also implies that one can detect non-trivial involutions on bimonoidal categories $\R$, which are not discrete.
First reduce the category $\R$ to the rig $\pi_0(\R)$, then group complete to $R=(-\pi_0(\R))\pi_0(\R)$. This is a ring and hence might give a
starting point for tools developed in algebraic $K$-theory of rings (as for example the trace map (cf.  \ref{tracing})).

\rem{An explicit example of that may be found in the proof of Proposition 8.1 in \cite{richter2010involution}, where Richter proves that the involution on
algebraic $K$-theory of complex topological $K$-theory $ku$ is non-trivial. In ``Two-Vector Bundles and Forms of Elliptic Cohomology'' \cite{baas2004two} the
authors construct a bimonoidal category  $\mathcal{V}$, which satisfies $K(ku)\simeq K(\mathcal{V})$ by the results of
``Ring Completion of Rig Categories''\cite{baas2009ring} and ``Stable Bundles over Rig Categories''\cite{baas909stable}. This interpretation allows
to give a map $K(ku) \rightarrow K^f(\Z)$ (where $K^f(\Z)$ is just a minor modification of $K(\Z)$) and this map is induced by a map of bimonoidal
categories $\mathcal{V}\rightarrow \mathcal{\R_\Z}$.

The $K$-theory of $\Z$ is, although or maybe because it is still not completely known, intensively studied. In particular there is a result of Farrell and Hsiang (Lemma 2.4
in \cite{farrell1rational}) that the involution on $K_*(\Z)$ is non-trivial, where the involution is induced by the identity. This result now implies that the involution
on $K(ku)$ is non-trivial.

Indeed the actual Proposition 8.1 in \cite{richter2010involution} also states the non-triviality of the involution on $K(ko)$ for $ko$ the spectrum of real topological
$K$-theory and the non-triviality of the involution on Waldhausen's $A$-Theory of the double classifying space of an abelian group $A(BBG)$. The proof proceeds
by an analogous strategy to identify $K(ko)$ and $A(BBG)$ as homotopy-equivalent to $K$-theory spaces of bipermutative categories, which also have a non-trivial
map to $\R_\Z$.

Of course a detailed investigation of all those structures would require knowledge of spectra, various identifications of $K$-theory and other techniques,
which are beyond the scope of this diploma thesis.}