\chapter{Introduction to $K$-Theory of Rings}
In order to investigate $K$-theory of rings with involution I first want to give a solid foundation of
$K$-theory of general rings. My approach will diverge drastically from the historical development in an effort
to homogenise the appearance of $K$-theory throughout this diploma thesis.

But in order to do the actual history justice, let me give some remarks on the history of (algebraic) $K$-theory. The $K$ in the designation $K$-theory is
reminiscent of Grothendieck's approach to the Riemann-Roch Theorem in which he studied coherent sheaves over an algebraic variety and looked
at classes - which is ``Klassen'' in German - of those sheaves modulo exact sequences in 1957. This approach has been published by Borel and Serre
in ``Le th\'eor\`eme de Riemann-Roch'' \cite{borelserre1958} (1958). This first group is what is nowadays called $K_0$.

About seven years later, Bass provided a definition of the group he called $K_1(A)$ for $A$ a ring and provided an exact sequence
$$K_1(A,\mathfrak{q})\rightarrow K_1(A) \rightarrow K_1(A/\mathfrak{q}) \rightarrow K_0(A,\mathfrak{q})\rightarrow K_0(A)\rightarrow K_0(\mathfrak{q})$$
in the paper ``$K$-Theory and Stable Algebras'' \cite{bass1964k} (1964). Since this exact sequence was built upon to define higher $K$-theory,
the name $K_1$ for this group is still the common one.

The next group $K_2$ was defined by Milnor in 1968 according to Bass' ``$K_2$ and symbols'' \cite{bass1969k} and again had an exact sequence connecting
it to $K_1$, along with a pairing $$K_1(A)\otimes K_1(A) \rightarrow K_2(A)$$ which is bilinear and antisymmetric, thus looked like a segment of a graded ring.
Since the exact sequence, this pairing and further relations look far too natural to be a coincidence, one was looking for a general sequence of $K$-groups $K_n(R)$,
which could be associated to a ring and which would yield a more structural explanation of the known results.

This was given by Milnor in ``Algebraic $K$-Theory and Quadratic Forms'' \cite{milnor1970algebraick}, but Milnor himself described
his definition as ``purely ad hoc'' (in \cite{milnor1970algebraick} as well). Furthermore Milnor's extension is restricted to be $K$-theory of fields, otherwise it would
not agree with the first three known $K$-groups.

The next attempt at defining $K$-theory for each natural number was given by Quillen in 1973 in his paper ``Higher Algebraic K-Theory: I'' \cite{quillen1973higher}.
Quillen defines a topological space, proves that its homotopy groups agree with the known definitions and derives some structural results, which translate algebraic relations
into topological relations between these newly defined spaces. These are the $K$-groups that are studied in this diploma thesis.

\section{$K$-Theory of Rings}
I will generally assume rings to be unital but not necessarily commutative rings.

As said before, I will deviate from the historical viewpoint and just define the $K$-theory space in order to give a more linear
concise summary for $K$-theory of rings. In preparation for that there are some necessary prerequisites in group homology.

\subsection{Classifying Space of a Group}
The first investigation focuses on a space $|BG|$, which arises in the context of classifying
principal $G$-bundles for a group $G$. It yields the result that isomorphism classes of $G$-bundles over a space $X$
are classified by homotopy classes of maps from $X$ to $|BG|$. It is relevant in the context of this thesis however
because of its homotopy groups (cf. Lemma \ref{classif}), so I will not elaborate on bundles any further.
For simplicial methods consult Loday's Appendix B in ``Cyclic Homology''\cite{LCy}, the ``basic
definitions'' in Chapter I of Goerss-Jardine ``Simplicial homotopy theory'' \cite{goerss2009simplicial}
and May's ``Simplicial Objects in Algebraic Topology''\cite{may1992simplicial}.

\defn{
For $G$ a group define a simplicial set as follows:
\begin{itemize}
 \item The set of $n$-simplices is given as $$BG_n:=G^n = G\times \ldots \times G,$$ i.e. $n$-simplices are $n$-tuples of elements of $G$,
 \item The face maps $d_i\colon BG_n \rightarrow BG_{n-1}$ are given by the equations
$$d_i(g_1,\ldots,g_n):=\begin{cases}
                       (g_2,\ldots,g_n) & \mathrm{~~~for~~} i=0\\
                       (g_1,\ldots,g_ig_{i+1},\ldots,g_n) & \mathrm{~~~for~~} i = 1,\ldots, n-1\\
                       (g_1,\ldots,g_{n-1}) & \mathrm{~~~for~~}i = n.
                       \end{cases}
$$
\item The degeneracies $s_i\colon BG_n \rightarrow BG_{n+1}$ are given by the equations
$$s_i(g_1,\ldots,g_n):=(g_1,\ldots,g_{i-1},1,g_i,\ldots,g_n)  \mathrm{~~~for~~} i=0,\ldots,n $$
which are putting a $1$ next to each component including the left- and rightmost position.
\end{itemize}

This constitutes a simplicial set (cf. May's ''Concise Course in Algebraic Topology'' \cite[Chapter 16, Section 5]{may1999concise}). \label{topbarg}
}
\prop{\cite[Section 16.5]{may1999concise}\label{classif} (1) Let $G$ be a topological group, then each homotopy group of the realisation of $BG$ is identified as follows
$$\pi_n(|BG|)\cong \pi_{n-1}(G) ~~(n \geq 1).$$
(2) If in particular $G$ is discrete, this implies
$$\pi_n(|BG|)=\begin{cases}
             G & ~~~\mathrm{for~~} n=1\\ 0 & \mathrm{~~~otherwise.}
            \end{cases}$$
\phantom{blubbel}\hfill$\Box$
}

\rem{
In particular this proposition answers the question whether each discrete group can be realised as a fundamental group affirmatively.
}

\subsection{Bar Construction on the Group Ring}
There is a simplicial abelian group closely related to the simplicial set discussed before.

\defn{\label{AbBar}
Let $G$ be a discrete group and $\Z\lbrack G\rbrack$ its group ring with integer coefficients.
Define the following simplicial abelian group:
\begin{itemize}
 \item Its group of $n$-simplices consists the free abelian group on $n$-tuples of $G$
$$B_n(G):=\Z\lbrack G\rbrack ^{\otimes n} \cong \Z\lbrack G^n \rbrack$$
 \item Its face and degeneracy maps are given by the linear extension of the respective maps given before in definition \ref{topbarg}.
\end{itemize}
The chain complex associated to this simplicial abelian group is called the bar complex associated to the group $G$.
}

\rem{
The common notations $BG$ for both objects are quite unfavourable, but on the other hand from
the simplicial perspective legitimate. I will denote the simplicial set with $BG_\bullet$ and
the simplicial abelian group with $B_*(G)$. The context in which they are used should provide sufficient grounds for just one interpretation to make sense.
}

The bar complex defines group homology.

\defn{\label{defgrphom}
Let $G$ be a group, then the group homology of $G$ is defined to be the homology
of the bar complex $G$, i.e. $$H_n(G):=H_n(B_*(G)).$$
}

The following result shows how well the two bar constructions interact:

\thm{\cite[Theorem 6.10.5]{weibel1995introduction} \label{alleshomologie}
For $G$ a group, $BG_\bullet$ the associated simplicial set and $B_*(G)$ its associated bar complex,
the singular homology with $\Z$-coefficients of $|BG|$ and group homology of $G$, which is homology of $B_*(G)$, are isomorphic, i.e.
$$H_n(|BG|,\Z) \cong H_n(G) = H_n(B_*(G)).$$
\phantom{blubbel}\hfill$\Box$
}

\rem{
I will use this isomorphism in quite an inexplicit manner, since chains in $B_*(G)$ and the cellular chain complex of $|BG|$
are isomorphic in a natural way. A very elegant proof via singular chains can be found in Weibel's ``An Introduction to Homological Algebra'' \cite{weibel1995introduction} Theorem 6.10.5.
}

There is an interpretation of the first homology group as follows:

\prop{\cite[Theorem 6.1.11]{weibel1995introduction} \label{h1}
For each group $G$, the first homology group of $G$ is naturally identified with the abelianised group
$$H_1(G) = G_{ab}.$$
\phantom{blubbel}\hfill$\Box$
}

\subsection{Plus-Construction on Classifying Spaces}
Since by lemma \ref{classif} the classifying space of a discrete group only provides one non-trivial homotopy group, the space has to
be modified in order to get $K_n(R) = \pi_n(KR)$ coinciding with the old definitions of $K_1$ and $K_2$. Furthermore
it ought to represent arithmetic information about the ring $R$. Therefore it would be quite counterintuitive to expect the $K$-groups in all
higher degrees to be trivial. Since the first three groups measure the linear algebra over a ring, it seems natural to approach the definition
of $K$-theory by looking at the classifying space of general linear groups over a ring.

In order to organise the linear groups of a ring into one structure, define the index category $\mathcal{N}$ with objects $\lbrack n \rbrack := \{1,\ldots,n\}$
and morphisms $f\colon \lbrack n \rbrack \rightarrow \lbrack n+k\rbrack$ given by $f(i) = i$ for each $i\in \lbrack n \rbrack$.

\defn{\label{stableGL}
For $R$ a unital ring and $F\colon \mathcal{N} \rightarrow Grp$ the functor given on objects by $$F(\lbrack n\rbrack) := GL_n(R)$$
and on morphisms by $$(F(f\colon \lbrack n \rbrack \rightarrow \lbrack n+1 \rbrack)(A))_{i,j} = \begin{cases}
                                                                 A_{f^{-1}(i),f^{-1}(j)} & ~~~~ i,j\in f(\lbrack n \rbrack)\\
								 0 & ~~~~(i \notin f(\lbrack n \rbrack) \vee j \notin f(\lbrack n \rbrack)) \wedge i\neq j\\
								 1 & ~~~~i=j \wedge i\notin f(\lbrack n \rbrack),
                                                               \end{cases}$$
define the stabilised linear group of $R$ as the colimit over $F$
$$GL(R) := \mathrm{colim}_\mathcal{N} F.$$
In this case it is evidently even a filtered colimit $GL(R):=\underrightarrow{\lim} GL_n(R)$.
}

\rem{
In less formal terms the stabilised linear group allows to identify each invertible matrix as a top left finite submatrix of an infinite
matrix, which otherwise has unit entries on the diagonal and zeroes everywhere else.
}

Defining the $K$-theory space of a ring involves studying the subgroup of elementary matrices in $GL(R)$. This was initially motivated by
Bass via $K_1$ in \cite{bass1964k}. Recall the following definition:

\defn{
Denote by $e_{i,j}(\lambda)$ the matrix with the following components
$$(e_{i,j}(\lambda))_{k,l}:=\delta_{k,l}+\lambda\delta_{i,k}\delta_{j,l}~~~~ i\neq j$$
That is $e_{i,j}(\lambda)$ is defined to be the identity matrix with exactly one off-diagonal component $\lambda\in R$.

The group of $n\times n$-elementary matrices $E_n(R)$ is defined as the subgroup of $GL_n(R)$ generated by $e_{i,j}(\lambda)$ for each $i\neq j$ and each $\lambda\in R$.

Since this is compatible with the stabilisation given in definition \ref{stableGL}, the stabilised group of elementary matrices can be defined in the same
fashion as in \ref{stableGL} by $$E(R):=\underrightarrow{\lim}E_n(R).$$ It is a subgroup of $GL(R)$ in a natural manner.
}

The next lemma is essential in constructing the $K$-theory space $K(R)$.

\lemma{\cite[Proposition 1.5]{srinivas1996algebraic} \label{Eperfect}
The stabilised elementary matrices generate a normal subgroup of the stabilised general linear group for any ring
$$E(R)\vartriangleleft GL(R),$$
which is perfect, i.e.
$$E(R)=\lbrack E(R), E(R)\rbrack.$$
Furthermore the commutator subgroup $\lbrack GL(R), GL(R) \rbrack$ of the stabilised general linear group is equal to the stabilised subgroup generated by elementary matrices, i.e.
$$E(R)=\lbrack GL(R), GL(R)\rbrack.$$
\hfill$\Box$
}

The following result gives the classical construction of Quillen to define higher $K$-theory:

\thm{\cite[Theorem 2.1]{srinivas1996algebraic} \label{plus}
Let $X$ be a CW-complex and $N\vartriangleleft \pi_1(X)$ a perfect normal subgroup of the fundamental group of $X$. Then there is a space $X^+$ with a
map $i\colon X \rightarrow X^+$, which are called the plus-construction on $X$ with respect to $N$. These satisfy the following properties:
\begin{enumerate}
 \item The map $i\colon X \rightarrow X^+$ induces the canonical projection on fundamental groups $$i_*\colon \pi_1(X) \rightarrow \pi_1(X^+)=\pi_1(X)/N.$$
 \item Each map $f\colon X \rightarrow Z$, which is trivial on the given subgroup $N$, that is $$\pi_1(f)\circ (j\colon N \rightarrow \pi_1(X))=0,$$ extends to a map on the plus-construction $\bar f\colon X^+ \rightarrow Z$.
The extension is unique up to homotopy in making the diagram
$$\xymatrix{
X \ar[dr]_{f} \ar[rr]^{i}&&X^+ \ar[dl]^{\bar f}\\
&Z & \\
}$$
commute up to homotopy.
 \item For each system of local coefficients $\mathcal{L}\colon \Pi_1(X^+) \rightarrow Ab$ on $X^+$ the induced map of the inclusion
$$i_*\colon H_n(X,i^*\mathcal{L})\rightarrow H_n(X^+,\mathcal{L})$$
induces an isomorphism on singular homology with local coefficients for each $n\geq 0$.
\end{enumerate}
\phantom{blubbel}\hfill$\Box$
}

\rem{
Let me emphasise some facts that the proof of this result yields. It is essential that the actual construction can be given by attaching $2$-cells which
precisely take care of the subgroup $N$ and additional $3$-cells to correct the defect on homology the $2$-cells might have caused. In that manner
it is legitimate to think of $i\colon X \rightarrow X^+$ as an inclusion.

I will not go into detail about local coefficients, but state that \ref{plus}.(3) implies $H_n(X,G)\cong H_n(X^+,G)$
for each abelian coefficient group $G$ understood as constant coefficients as well, specifically for $G=\Z$.

Property $(2)$ in particular implies that any two plus-constructions to a fixed perfect normal subgroup are homotopy-equivalent by the usual argument.
}

The results \ref{Eperfect} and \ref{plus} combine to Quillen's definition of higher $K$-theory.

\defn{
For $R$ a ring define the $K$-theory space of $R$ to be the plus-construction on the classifying space of $GL(R)$ with respect to the elementary matrices $E(R)$, i.e.
$$K(R) := |BGL(R)|^+$$
and for $n\geq 1$ define the $K$ groups of $R$ by
$$K_n(R):=\pi_n(K(R))~~~ (n\geq 1).$$
}

\rem{Be aware that I completely avoid $K_0(R)$ here and in all the diploma thesis, because
it is exceptional in most cases.
}

As a defining property of the plus construction there are immediate reinterpretations of $K_1$:

\prop{\cite[Theorem 2.1 and Proposition 1.5]{srinivas1996algebraic} There are the following natural identifications\label{k1gl}
$$K_1(R)=GL(R)/E(R) = GL(R)/\lbrack GL(R), GL(R)\rbrack = GL(R)_{ab} = H_1(GL(R)).$$
\begin{proof}
The first equality is the definition of $K(R)$, the second identification is a consequence of lemma \ref{Eperfect}, the third is the usual identification giving
a natural model for the abelianised group of any group and the last identification is the classical one already cited in proposition \ref{h1}.
\end{proof}
}

I will only refer to \cite{srinivas1996algebraic} again to note that there are interpretations for $K_2$ and $K_3$ as well, which are not used in this
diploma thesis.

The identifications given above in particular yield a useful tool for commutative rings.

\prop{\cite[Theorem 2.2.1]{rosenberg1994algebraic}\label{det}
For $R$ a commutative ring, the determinant maps for each finite degree $\det_n\colon GL_n(R) \rightarrow R^\times$ stabilise to give a
map $$\det\colon GL(R) \rightarrow R^\times,$$ which is a group homomorphism with $$\det(e_{i,j}(\lambda))=1_R$$
for each $i\neq j$ and $\lambda \in R$. In particular it satisfies $E(R)\subset \ker(det\colon GL(R)\rightarrow R^\times)$ and hence factors as follows
$$
\xymatrix{
GL(R)\ar[dr]\ar[rr]^{\det}&& R^\times\\
&GL(R)/E(R)=K_1(R).\ar[ur]
}
$$
\phantom{blubbel}\hfill $\Box$
}

It is thus legitimate to write $\det\colon K_1(R)\rightarrow R^\times$ as well and to call it the determinant map as well.
This is useful because there is an obvious inclusion $j\colon R^\times = GL_1(R) \rightarrow GL(R) \rightarrow GL(R)/E(R)$,
which yields $\det\circ j = id_{R^\times}$ and hence gives a natural splitting exact sequence
$$0\rightarrow SK_1(R) \rightarrow K_1(R) \rightarrow R^\times \rightarrow 0,$$
which in particular implies
$$K_1(R)\cong SK_1(R)\oplus R^\times$$
for $SK_1(R) := SL(R)/E(R)$, where $SL(R)$ is the usual special linear group stabilised as $GL(R)$ and $E(R)$ before. For commutative
rings $R$ the problem of computing $K_1(R)$ thus reduces to computing units --- which may be a very hard problem (cf. Chapter 4) --- and computing the reduction
of matrices of determinant one by elementary matrices --- which is hard in general as well, but can be feasible.

\rem{\label{HurewiczKH}
Let me emphasise that there is a natural map $K_i(R) \rightarrow H_i(GL(R))$ given as follows:

The $K$-theory of a ring is defined as $K_i(R):= \pi_i(|BGL(R)|^+) (i\geq 1)$ and the plus-construction does not change homology
$$H_i(|BGL(R)|^+,\Z)\cong H_i(|BGL(R)|,\Z) = H_i(GL(R)).$$
Hence the Hurewicz-homomorphism gives a natural map $$K_i(R) = \pi_i(|BGL(R)|^+) \rightarrow H_i(GL(R))$$
and specifically in degree $1$ this can be understood as the identity on $GL(R)/E(R)$, since there are canonical
identifications given in proposition \ref{k1gl} $$H_1(GL(R))=GL(R)_{ab}=GL(R)/\lbrack GL(R),GL(R)\rbrack = GL(R)/E(R)=K_1(R).$$}