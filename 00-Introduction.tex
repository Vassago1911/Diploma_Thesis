\chapter*{Introduction}
Algebraic $K$-theory of bimonoidal categories arises as a structure of interest in ``Two-vector bundles and forms of
elliptic cohomology'' \cite{baas2004two} by Nils Baas, Bj\o{}rn Ian Dundas and John Rognes and its follow-up papers
``Ring completion of rig categories'' \cite{baas2009ring} and ``Stable bundles over rig categories''\cite{baas909stable}
by these three authors joint with Birgit Richter. (These papers formerly were the one paper ``Two-vector bundles define a form of elliptic cohomology''.)
Specifically they investigate the category of two-vector-bundles. By analogy to the case of principal bundles
over a topological space they study two-vector-bundles by examining a represented functor $\lbrack X, K(\mathcal{V})\rbrack$, where
$K(\mathcal{V})$ is the algebraic $K$-theory spectrum of the bimonoidal category of finite-dimensional complex vector spaces $\mathcal{V}$.
This furthermore embeds into the context of interpolating between the complexity of singular homology, which only captures few phenomena, and
the complexity of complex bordism, which detects all levels of periodic phenomena at once. The cohomology theory defined by $K(\mathcal{V})$
is in a precise manner one level more complex than topological $K$-theory is.

The topic of involutions is introduced by the fact that it would be negligent to ignore that the category of complex vector spaces
is equipped with an involution naturally induced by conjugation in complex numbers.\\[3ex]

This diploma thesis studies the $K$-theory and Hochschild homology of rings with involution.
In detail the diploma thesis arose from the motivation to examine non-trivial involutions on $K$-theory of bimonoidal categories by studying
non-trivial involutions on rings. To that end after introducing the basic concepts I define the algebraic $K$-theory of
strict bimonoidal categories following Birgit Richter's ``An involution on the $K$-theory of
bimonoidal categories with anti-involution''\cite{richter2010involution}. Furthermore if an object has an associated
anti-involution, this gives an associated involution on its $K$-theory, which we define according to \cite{richter2010involution} as well.

As an example I explicitly present $K$-theory and involutions of group rings and more specifically Laurent polynomials.
These examples illustrate the limitations that involutions can be studied on $K$-groups directly with the same difficulty
that $K$-groups can be computed. There is a need for further tools.\\[1ex]

The last chapter is the heart of this diploma thesis. I investigate the Dennis trace map
$$\mathrm{Dtr}\colon K_n(R) \rightarrow HH_n(R)$$
and the map associated to an anti-involution on Hochschild homology.
The main result \ref{derKernueberhaupt} of this diploma thesis is that the trace map $\mathrm{Dtr}$ does commute with the induced involutions.
Thus the trace map provides an additional tool to study involutions on $K$-theory and can help to prove non-triviality.
The result should be compared to the statement by Bj\o{}rn Ian Dundas in the introduction of \cite{dundas2000cyclotomic} that his
functorial definition of a trace map in particular implies that it respects involutions on $K$-theory and topological Hochschild homology.
This diploma thesis serves as an algebraic analogue of a fact known on the topological level, although it is not directly implied by that.

Finally I discuss the example of the integers with an adjoined prime root of unity $\Z\lbrack \zeta_p \rbrack$ and discuss the non-triviality
in that case in contrast to Laurent polynomials, which goes to show the usefulness of the Dennis trace map as well as its defects.\\[2ex]

This text is organised as follows:
The first two chapters are dedicated to study algebraic $K$-theory in the standard context of rings and specifically study rings with involution by
providing an induced involution on the $K$-groups.
The third chapter essentially is a summary of Birgit Richter's ``An involution on the $K$-theory of
bimonoidal categories with anti-involution''\cite{richter2010involution}. Furthermore the relations between
a strict bimonoidal category and its associated ring embed this paper into the non-triviality statements of the
following chapters.
I provide examples of rings with families of involutions in Chapter 4. All of these examples have non-trivial involutions on $K_1$.
Finally, chapter 5 introduces the Dennis trace map as an example of a useful tool which can be extended to the context
of rings with involution and provide examples to evaluate its usefulness in the ring context.

\section*{Acknowledgements}
Let me take this opportunity to thank several hands full of people, who were directly or indirectly involved in this diploma thesis:

Of course my primary thanks are addressed at Birgit Richter for suggesting a topic that allowed to explore a lot of mathematics, lots of which did
not make it into this thesis, and giving me the freedom to do so, which has taken a considerably bigger amount of time than it usually would :)

Furthermore in an order, where each place is equivalent to first place, I thank:

Stephanie Ziegenhagen, for loads of non-mathematical and mathematical discussions alike, the inspiration of your strict, thorough, clear head and
warm, just and honestly outspoken heart is of ever-growing and indescribable importance to me. I agree: May the force of universal properties save us
from coordinates!

Hannah K\"onig for giving me the quickest introduction known to mankind to my course of studies about five years ago, thus saving generations of kids from my
school teaching :)

Thomas Nikolaus for allowing me to pick your head every once in a while to provide another view on mathematics, university or other things as a whole.

The founders of the internet for giving me access to invaluable things such as Google, Google scholar, dict.leo.org, Ubuntu, YouTube, Allen Hatcher's
``Algebraic Topology'', the arXiv and lots of other distracting or helpful things.

Donald Knuth and numerous other persons for \LaTeX.

Christoph Schweigert for several helpful discussions, actions, courses and in particular a motivating starting point in ``Linear Algebra''. Quote: ``Linear
Algebra organises the brain enormously.'' Today I know your influence on that course strengthens that effect tremendously.

Astrid D\"orh\"ofer and Eva Kuhlmann for warming up our floor by pure presence and for non-bureaucratic help whenever needed.


Of course there are people remaining outside the Geomatikum as well:

Mark Wroblewski and Raluca Oancea for putting up with my peculiarities for all those years - stress-induced and not stress-induced alike. Additional
thanks for giving me an occasional kick, whenever I might need it, for sports, spare time, games, discussions or just general fun :) and of course
just being the wonderful factor of my life you each on your own and together are.

Ute Zwicker for providing yet another view on life, which helped to keep me sane lots and lots of times, for just general killing of time on the train by
discussions, which could have happily went on for more hours than any train could ever be delayed.

Last but not least I thank my parents for providing for me - financially and otherwise - for all that time, although I am aware that I can be an expensive
beast regarding money, nerves and other things every once in a while or permanently :)

All of you should know that this diploma thesis would not exist without you.