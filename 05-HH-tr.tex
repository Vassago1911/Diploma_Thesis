\chapter{Hochschild Homology of Rings}
One of the tools to detect non-trivial classes in $K$-theory is the Dennis trace map from $K$-theory to Hochschild homology. In the following chapter
I will check that this map commutes with the induced involutions on the groups in question, so that the Dennis trace map is a detection tool for rings with
involution as well.

So at least by the formal similarity between Hochschild homology and Topological Hochschild homology visible in  Chapter IX, Definition 2.1
of ``Rings, Modules and Algebras in stable homotopy theory'' \cite{elmendorf2007rings} this algebraic analogue should not come as a surprise.

\section{Definition of Hochschild Homology}
I will consider Hochschild homology restricted to the case of unital $\Z$-algebras only, more commonly known as rings.
In this I am mainly following the chapters 1 and 8.4 of \cite{LCy}.

For $R$ a unital ring, let $C_n(R):=R^{\otimes n+1}$ and let
$$d_i^n\colon C_n(R) \rightarrow C_{n-1}(R) ~~~\mathrm{for ~~~} i = 0,\ldots,n$$
be defined as
$$d_i^n ( r_0 \otimes r_1 \otimes \ldots \otimes r_n ) = \begin{cases} r_0\otimes \ldots \otimes r_ir_{i+1} \otimes \ldots \otimes r_n & ~~ i= 0,\ldots,n-1\\
                                                                                       r_nr_0 \otimes \ldots \otimes r_{n-1} & ~~ i=n \end{cases}$$
and furthermore
$$s_i^n\colon C_n(R) \rightarrow C_{n+1}(R) ~~~\mathrm{for ~~~} i = 0,\ldots,n$$
is
$$s_i^n (r_0 \otimes r_1 \otimes \ldots \otimes r_n) = r_0\otimes \ldots \otimes r_i \otimes 1 \otimes r_{i+1} \otimes \ldots \otimes r_n.$$
Note that no $s_i$ places the unit on the left of $r_0$.

This constitutes a simplicial module $C_*(R)$ and hence allows to associate a chain-complex with boundary
map $d^n := \sum\limits_{i=0}^n(-1)^i d_i^n$ and find a subcomplex of degenerate elements via:
$$D_n(R):=\sum\limits_{i=0}^{n-1} \mathrm{im} s_i^{n-1}$$
Then the following is a standard fact about simplicial modules:

\prop[Normalising modules \lbrack cf. Loday Prop. 1.6.5\rbrack]{The canonical projection $$C_* \rightarrow C_*/D_*$$ is a quasi-isomorphism, i.e. it induces an isomorphism
of homology groups.\label{normal}}
\defn{For $R$ a unital ring the Hochschild-homology of $R$ is defined as the homology of the chain complex associated to $C_*(R)$ given above:
$$HH_n(R):=H_n(C_*(R)) ~~~~~n \in \N_0.$$}

\rem{The preceding proposition thus allows the calculation of Hochschild-homology via $C_*(R)/D_*(R)$ as well.}

Since I focused a lot on group rings so far it is natural to investigate Hochschild homology of group rings as well. For group rings
with integer coefficients there is a very satisfactory identification of Hochschild homology.

\thm{(\cite{LCy}, Proposition 7.4.2)\label{HH=h}
Let $G$ be a group, then the Hochschild homology of the group ring $\Z\lbrack G\rbrack$ is naturally isomorphic to the group homology of $G$ (cf. Theorem \ref{alleshomologie})
$$HH_*(\Z\lbrack G\rbrack) \cong H_*(G).$$
\phantom{blubbel}\hfill$\Box$
}

\ex{
This result and proposition \ref{h1} in particular give a natural identification of the first Hochschild homology group,
which is of particular interest for the following examples $$HH_1(\Z\lbrack G\rbrack) \cong H_1(G) = G_{ab}.$$
}

Group rings might seem like an artificial class of examples, but they determine the first Hochschild homology
of $\Z\lbrack \zeta_p\rbrack$ for $\zeta_p$ a root of unity to a prime number $p\geq 3$. For convenience I first identify
this ring as a quotient of $\Z\lbrack X\rbrack$.

\prop{
For $p\in \N$ a prime number, $\zeta_p\in \mathbb{C}$ a $p$-th root of unity and
$$\varphi_p(X)=\sum\limits_{i=0}^{p-1} X^{i}\in \Z\lbrack X \rbrack \subset \Q\lbrack X\rbrack$$
the cyclotomic polynomial of degree $p$, there is an isomorphism
$$\Z\lbrack X \rbrack / (\varphi_p) \cong \Z\lbrack \zeta_p \rbrack$$
induced by the evaluation homomorphism $X \mapsto \zeta_p$.
\begin{proof}
This fact is essentially what one would expect from the case of field extensions, because there is an isomorphism $\Q\lbrack X \rbrack / (\varphi_p) \cong \Q\lbrack \zeta_p \rbrack$ given
by identification of $\varphi_p$ as the minimal polynomial of $\zeta_p$. Nonetheless there is a technical point to show that
the principal ideal generated in $\Q\lbrack X\rbrack$ by $\varphi_p$ restricts to the principal ideal generated by $\varphi_p$ in $\Z\lbrack X\rbrack$. Formally this is
$$(\varphi_p)_{\Q\lbrack X\rbrack}\cap \Z\lbrack X \rbrack \subset (\varphi_p)_{\Z\lbrack X \rbrack},$$
i.e. one cannot generate more polynomials with integer coefficients, if one allows polynomials with rational coefficients. This is a consequence
of the fact that the coefficients of the cyclotomic polynomial for a prime degree are all $1$, which is a unit in $\Z$.
\end{proof}}

\rem{
There is the elementary identification $$\Z\lbrack \Z/p\Z \rbrack \cong \Z\lbrack X \rbrack/(X^p - 1)$$ and hence a projection
$$\pi\colon \Z\lbrack \Z/p\Z \rbrack \cong \Z\lbrack X\rbrack /(X^p - 1) \longrightarrow \Z\lbrack X\rbrack /(\varphi_p) \cong \Z\lbrack \zeta_p \rbrack,$$
because of the factorisation $X^p - 1 = \varphi_p(X)(X-1).$\\
In the next proof I will use both descriptions of $\Z\lbrack \zeta_p\rbrack$ as either a subring of $\mathbb{C}$ or as a quotient of $\Z\lbrack X\rbrack$ without explicit
mention of the isomorphism.}

I use this projection to determine the Hochschild homology of $\Z\lbrack \zeta_p \rbrack$ in degree $1$.

\thm{\label{degreeiso}
The first Hochschild homology group of the integers with an adjoint $p$-th root of unity $\zeta_p$ for an odd prime number $p$ is given by
$$HH_1(\Z\lbrack \zeta_p \rbrack) \cong \Z/p\Z$$
with the isomorphism given on generators by $\lbrack \zeta_p^i\otimes \zeta_p^j \rbrack \mapsto \lbrack j \rbrack$.

\begin{proof}
First check that $HH_1(\Z\lbrack \zeta_p \rbrack)$ is finite. To that end study the chain map induced by $\pi$ in degrees $0$ and $1$
$$\xymatrix{
\ldots \ar[r] & \Z\lbrack \Z/{p\Z} \rbrack \otimes \Z\lbrack \Z/{p\Z} \rbrack \ar[r]^{~~~~~~~~d^1} \ar[d]^{\pi_*} & \Z\lbrack \Z/{p\Z} \rbrack  \ar[r] \ar[d]^{\pi_*=\pi} & 0\\
\ldots \ar[r] & \Z\lbrack \zeta_p \rbrack \otimes \Z\lbrack \zeta_p\rbrack \ar[r]^{d^1}                     & \Z\lbrack \zeta_p \rbrack \ar[r]              & 0.
}$$
Evidently $\pi_*$ is still surjective in each degree, in particular in degree $1$. But since both of the involved rings are commutative,
the first boundary map is trivial $d^1=0$; so $\pi_*$ is surjective on cycles as well and thus also on homology. Hence the first Hochschild homology
group $HH_1(\Z\lbrack \zeta_p\rbrack)$ is a quotient of $\Z/p\Z$ and thus either trivial or $\Z/p\Z$ itself.

Consider the following map
$$\begin{aligned}
   \Phi\colon \Z\lbrack X\rbrack /(\varphi_p) \otimes \Z\lbrack X \rbrack / (\varphi_p) &\rightarrow \Z/p\Z\\
                      \lbrack X^i \rbrack \otimes \lbrack X^j\rbrack &\mapsto \lbrack j \rbrack
  \end{aligned}$$
It is well-defined with respect to the quotient by $\varphi_p$ in either component by the following calculations.
In the first component this is
$$\Phi\left(\left[ \sum\limits_{i=0}^{p-1} X^i \right] \otimes \left[ X^j \right]\right) = \sum\limits_{i=0}^{p-1} \lbrack j \rbrack
 = \lbrack p\cdot j \rbrack = 0.$$
In the second component the relation introduced by $\varphi_p$ gives
$$\Phi\left(\left[ X^j\right] \otimes \left[ \sum\limits_{i=0}^{p-1} X^i \right]\right) = \left[ \sum\limits_{i=0}^{p-1} i \right] = \left[\frac{p(p-1)}{2}\right] = 0,$$
because $(p-1)$ is even, which implies that $\frac{p(p-1)}{2}$ is divisible by $p$.

It is obvious that this map is surjective and furthermore on boundaries this yields

$$\begin{aligned}
\Phi(d^2(X^i\otimes X^j \otimes X^k))&=\Phi(X^{i+j}\otimes X^k-X^{i}\otimes X^{j+k} + X^{i+k}\otimes X^j) \\
 &= \lbrack k \rbrack - \lbrack j+k\rbrack +\lbrack j \rbrack=0.
  \end{aligned}
$$
So this map is still well-defined and surjective as a map from $HH_1(\Z\lbrack \zeta_p\rbrack)$ to $\Z/p\Z$ and
thus by the preceding calculation it is an isomorphism.
\end{proof}}

\section{Construction of the Trace Map}
A tool to detect non-trivial classes in $K$-groups is the so called Dennis trace map from $K$-theory to the just defined Hochschild homology of $R$.
It was defined by Keith Dennis, but that paper was never published, hence the first reference is Kiyoshi Igusa \cite{IgusaWhat} in 1984.

\rem{Recall the usual trace map of matrices
$$tr\colon M_r(R) \rightarrow R$$
with $tr(A) := \sum\limits_{i=1}^r A_{ii} $, which is obviously compatible with the stabilisation of square
 $r \times r$-matrices given by
$$
\begin{aligned}
i\colon  M_r(R) &\rightarrow M_{r+1}(R)\\
A&\mapsto \begin{pmatrix}
           A & 0_r\\ 0_r^t & 0
          \end{pmatrix},
\end{aligned}
$$ which just adds bordering zeroes.}

There is generalisation of the trace map to tensor products of matrix rings (of equal size) as follows:

\defn{
The generalised trace map
$$tr\colon M_r(R)^{\otimes n+1} \rightarrow R^{\otimes n+1}$$
is given by
$$tr(A^0 \otimes A^1 \otimes \ldots \otimes A^n):= \sum\limits_{i\in J} A^0_{i_0,i_1} \otimes A^1_{i_1,i_2} \otimes \ldots \otimes A^n_{i_n,i_0},$$
where $i=(i_0,\ldots,i_n)$ is from the set $J = \{1,\ldots, r \}^{n+1}$.}

\defn{
The fusion map
$$\mathrm{fus}\colon \Z\lbrack GL_r(R) \rbrack \rightarrow M_r(R)$$
is given by the extension of the identity on $GL_r(R)$ as follows
$$\mathrm{fus}\left( \sum_i k_i A_i\right) = \sum_i k_i A_i.$$}

\rem{
This might look quite tautological, but the point is to replace the formal sums in the group ring
of $GL_r(R)$ by actual sums in $M_r(R)$.
Be aware that $GL_r(R)$ has to stabilise by adding bordering zeroes and a $1$ on the diagonal
$$\begin{aligned}
   GL_r(R)&\rightarrow GL_{r+1}(R)\\
   A &\mapsto \begin{pmatrix}
              A & 0_r \\ 0_r^t & 1
             \end{pmatrix}.
  \end{aligned}
$$
So the fusion map evidently does not stabilise (cf. Loday \cite[8.4.1]{LCy}, The Fusion Map), but I will take care of that problem later.\\[2ex]}


\label{tracing}
Study the following sequence of maps (for $B_*$ the bar complex (cf. Definition \ref{AbBar}) and $C_*$ the Hochschild complex)
$$\xymatrix{
B_n(\Z\lbrack GL_r(R)\rbrack) \ar[r]^{\mathrm{inc}} &
 C_n(\Z\lbrack GL_r(R) \rbrack)  \ar[r]^-{C_n(\mathrm{fus})}
 &C_n(M_r(R)) \ar[r]^{~~C_n(\mathrm{tr})} &
 C_n(R) \ar[d]^-{p_n} \\ & & & C_n(R)/D_n(R)
}$$
where in sequence there are the following maps
\begin{itemize}
\item $\mathrm{inc} \colon Z\lbrack (GL_r(R))^n\rbrack \rightarrow Z\lbrack GL_r(R)\rbrack^{\otimes n+1}$ ~~ is the map given by $$\mathrm{inc}(g_1,\ldots,g_n) = (g_1\cdot \ldots \cdot g_n)^{-1}\otimes g_1 \otimes \ldots \otimes g_n,$$
\item $\mathrm{fus}^{\otimes n+1}\colon Z\lbrack GL_r(R) \rbrack^{\otimes n+1} \rightarrow M_r(R)^{\otimes n+1}$ ~~ is the $(n+1)$-fold tensor product of the fusion map,
\item $tr \colon M_r(R)^{\otimes n+1} \rightarrow R^{\otimes n+1}$ ~~ is the generalised trace map defined before,
\item $p_n \colon C_n(R) \rightarrow C_n(R)/D_n(R)$ ~~is the canonical projection seen before.
\end{itemize}

\lemma{Each of these is a simplicial map, hence a chain map on the induced chain complexes.

\begin{proof}
The fusion map evidently is a simplicial morphism, since it is just an extension of the identity on generators, and the identity is a simplicial map.
The canonical projection is simplicial as well, since the quotient is taken by a simplicial subcomplex.

For the inclusion of the bar complex into the Hochschild complex I check the interesting case $i=n$
$$
\begin{aligned}
(\mathrm{inc}\circ d_n(g_1\otimes \ldots\otimes g_n)) &= \mathrm{inc}(g_1\otimes \ldots\otimes g_{n-1})\\
&=(g_1\ldots g_{n-1})^{-1}\otimes g_1 \otimes \ldots \otimes g_{n-1}\\
&=g_n(g_1\ldots g_{n})^{-1}\otimes g_1 \otimes \ldots \otimes g_{n-1}\\
&=d_n((g_1\ldots g_n)^{-1}\otimes g_1 \otimes \ldots \otimes g_n)\\
&=d_n(\mathrm{inc}(g_1 \otimes \ldots \otimes g_n)),
\end{aligned}$$
which shows quite well, how the factor $(g_1\ldots g_n)^{-1}$ contributes to the inclusion, while the other relations are straightforward calculations.

For the trace map from the Hochschild complex on $r\times r$-matrices over $R$ to the Hochschild complex over $R$ itself,
check the exceptional case $i=n$ again
$$
\begin{aligned}
(d_n\circ tr)(A^0\otimes\ldots\otimes A^n) &= d_n\left(\sum\limits_{i\in J} A^0_{i_0,i_1}\otimes A^1_{i_1,i_2} \otimes \ldots \otimes A^n_{i_n,i_0}\right)\\
&= \sum\limits_{i\in J} A^n_{i_n,i_0}A^0_{i_0,i_1}\otimes A^1_{i_1,i_2} \otimes \ldots \otimes A^{n-1}_{i_{n-1},i_n}\\
&= \sum\limits_{i_0=1}^r\sum\limits_{i'\in \{1,\ldots,r\}^n}A^n_{{i'}_n,i_0}A^0_{i_0,{i'}_1}\otimes A^1_{{i'}_1,{i'}_2} \otimes \ldots \otimes A^{n-1}_{{i'}_{n-1},{i'}_n}\\
&= \sum\limits_{i'\in \{1,\ldots,r\}^n}(A^nA^0)_{{i'}_n,{i'}_1}\otimes A^1_{{i'}_1,{i'}_2} \otimes \ldots \otimes A^{n-1}_{{i'}_{n-1},{i'}_n} \\
&= tr(A^nA^0 \otimes A^1 \otimes \ldots \otimes A^{n-1})\\
&= (tr\circ d_n)(A^0 \otimes A^1 \otimes \ldots \otimes A^{n-1}\otimes A^n).
\end{aligned}
$$
The same argument at each index gives the other relations as well. Thus follows the claim that all the maps given above are maps of simplicial abelian groups.
\end{proof}}

So far I have only defined all this on matrices of fixed degrees.
Since the fusion map on its own does not stabilise, it is quite remarkable that the trace map does.

\thm[Dennis Trace Map]{\cite[Proposition 8.4.3]{LCy} The sequence of simplicial modules and maps defines a natural map (in unital rings $R$)
$$\overline{Dtr}\colon H_n(GL_r(R)) \rightarrow HH_n(R)$$
for all $n,r\in \N$, which is compatible with the inclusions $i_r\colon GL_r(R) \rightarrow GL_{r+1}(R)$ and as a consequence gives a natural map
$$\overline{Dtr}\colon H_n(GL(R)) \rightarrow HH_n(R)$$
which composed with the Hurewicz homomorphism $K_n(R)\rightarrow H_n(GL(R))$ is called the Dennis trace map $Dtr := \overline{Dtr}\circ h $.

\begin{proof}
This proof is directly taken from \cite[Proposition 8.4.3]{LCy}.

It is evident that all the factors are natural maps in unital rings and via application of the functor $H_n\colon \mathrm{sAb} \rightarrow \mathrm{Ab}$,
there is an induced map $\overline{Dtr}$. So the only point is to show that $\overline{Dtr}$ is stable. It is clear that
$\mathrm{inc} \colon Z\lbrack GL_r(R) \rbrack^{\otimes n} \rightarrow Z\lbrack GL_r(R)\rbrack^{\otimes n+1}$
is natural and stable, so focus on the remainder
$$Z\lbrack GL_r(R) \rbrack^{\otimes n+1} \rightarrow M_r(R)^{\otimes n+1} \rightarrow R^{\otimes n+1} = C_n(R) \rightarrow C_n(R)/D_n(R)$$
and calculate
$$(p\circ tr \circ \mathrm{fus} )\left( \begin{pmatrix} A^0 & \\ & 1\end{pmatrix} \otimes \begin{pmatrix} A^1 & \\
 & 1\end{pmatrix} \otimes \ldots \otimes \begin{pmatrix} A^n & \\ & 1\end{pmatrix} \right)$$
$$\begin{aligned}&= (p \circ tr) \left( \begin{pmatrix} A^0 & \\
& 1\end{pmatrix} \otimes \begin{pmatrix} A^1 & \\ & 1\end{pmatrix} \otimes \ldots \otimes \begin{pmatrix} A^n & \\ & 1\end{pmatrix} \right)\\
&=p\left(\sum\limits_{i\in J}\begin{pmatrix} A^0 & \\ & 1\end{pmatrix}_{i_0,i_1} \otimes \begin{pmatrix} A^1 & \\
& 1\end{pmatrix}_{i_1,i_2} \otimes \ldots \otimes \begin{pmatrix} A^n & \\ & 1\end{pmatrix}_{i_n,i_0}\right)\\\end{aligned}$$
If for any $j$ the index equals $i_j = r+1$, then the only case, in which $\begin{pmatrix} A^j & \\ & 1\end{pmatrix}_{i_j,i_{j+1}}$
can be non-zero, is when $i_{j+1}=r+1$ as well. Repeat this argument for each $j$, then the only non-trivial additional index
family $i$ is $i=(r+1,r+1,\ldots,r+1)$, each other index family can only contribute
non-trivial factors, if $i_j\leq r  ~~\forall j\in \{1,\ldots,r\}$. This recovers the trace map of the $GL_r(R)$ and gives the equality
$$=p\left(tr( A^0 \otimes A^1 \otimes \ldots \otimes A^n) + 1\otimes 1\otimes\ldots\otimes 1\right)$$
But $\ker p$ contains each element which has at least on unit entry in any position except for the first. This yields
$$=(p\circ tr)( A^0 \otimes A^1 \otimes \ldots \otimes A^n) = (p\circ tr\circ \mathrm{fus})( A^0 \otimes A^1 \otimes \ldots \otimes A^n).$$
Therefore $\overline{Dtr}$ is stable, if regarded as a map $B_n(\Z\lbrack GL_r(R)\rbrack) \rightarrow C_n(R)/D_n(R)$, so applying homology yields the induced map
$$\overline{Dtr}\colon H_n(GL(R))\rightarrow H_n(C_n(R)/D_n(R)) \stackrel{\mathrm{\ref{normal}}}{\cong}HH_n(R).$$
\end{proof}}

As a recurring example I specifically study degree $1$ of Hochschild homology for $\Z\lbrack \zeta_p \rbrack$ for $\zeta_p$ a $p$-th root of unity and $p$ prime.
So far I presented  that $K_1(R\lbrack G \rbrack)$ contains the units of $\Z\lbrack \zeta_p \rbrack$, which includes the powers of $\zeta_p$.
This gives the following result:

\thm{\label{nontrivroot}
The Dennis trace map $$Dtr\colon K_1(\Z\lbrack \zeta_p \rbrack)\rightarrow HH_1(\Z\lbrack \zeta_p \rbrack)$$ is non-trivial.
\begin{proof}
Consider $\zeta_p$ as a unit, then this gives the equality
$$Dtr(\zeta_p)=tr(fus(\zeta_p^{-1}\otimes \zeta_p))=\zeta_p^{p-1}\otimes \zeta_p,$$
which under the isomorphism of theorem \ref{degreeiso} yields
$$\Phi(Dtr(\zeta_p))=\Phi(\zeta_p^{p-1}\otimes \zeta_p) = \lbrack 1 \rbrack\in \Z/p\Z \cong HH_1(\Z\lbrack \zeta_p \rbrack),$$
so $Dtr$ is even surjective, in particular it is not the zero map.
\end{proof}}


\section{Involution on Hochschild Homology}
So far it is not clear, whether there is an induced involution on $HH_*(R)$, if there is one on $R$.
Again I am following Loday \cite{LCy}, specifically Section 5.2.

\subsection{Opposing the Simplicial Structure on the Hochschild Complex}
The following complex gives a comparison on how opposing the ring changes the simplicial structure of the Hochschild complex:

\defn{
For $\tilde{C}_n(R) := R^{\otimes n+1}$ set $$\tilde{d}_{i}:=d_{n-i} ~~~~\mathrm{and}~~~~~\tilde{s}_i := s_{n-i}.$$
This still is a simplicial module, call it the opposite Hochschild complex.\label{Hochopp}}

\lemma{(Adapted from \cite[5.2.1]{LCy}) \label{HH(op)} There is a natural isomorphism of simplicial modules $$C_*(R) \rightarrow \tilde{C}_*(R^{op})$$ given by $$w_{HH}(r_0\otimes r_1 \otimes \ldots \otimes r_n) = r_0 \otimes r_n \otimes \ldots \otimes r_1,$$
and $w_{HH}$ is a map of simplicial abelian groups.
\begin{proof}
This is a calculation what $w_{HH}$ does to face and degeneracy maps, where again I denote the opposed multiplication by $\circ$.

The map $w_{HH}$ commutes with face maps by the following calculation
$$
\begin{aligned}
w_{HH} &d_i(r_0\otimes r_1 \otimes \ldots \otimes r_n)=\begin{cases}w_{HH}(r_0 \otimes \ldots \otimes r_ir_{i+1} \otimes \ldots \otimes r_n) & 0\leq i < n \\
                                                               w_{HH}(r_nr_0 \otimes \ldots \otimes r_{n-1}) & i = n
                                                  \end{cases}\\
						&=\begin{cases}r_0 \otimes r_n\otimes \ldots \otimes r_ir_{i+1} \otimes \ldots \otimes r_1 & i = 0,\ldots,n-1 \\
                                                               r_nr_0 \otimes r_{n-1}\otimes \ldots \otimes r_1 & i = n
                                                  \end{cases}\\
						&=\begin{cases}r_0 \otimes r_n\otimes \ldots \otimes r_{i+1}\circ r_i \otimes \ldots \otimes r_1 & i = 0,\ldots,n-1 \\
                                                               r_0\circ r_n \otimes r_{n-1}\otimes \ldots \otimes r_1 & i = n
                                                  \end{cases}\\
						&=\begin{cases}d_{n-i}(r_0 \otimes r_n\otimes \ldots\otimes r_1) & i = 0,\ldots,n-1 \\
                                                               d_0(r_0 \otimes r_n \otimes \ldots \otimes r_1) & i = n
                                                  \end{cases}\\
&=\left.\begin{cases}d_{n-i}w_{HH}(r_0 \otimes \ldots\otimes r_n) & i = 0,\ldots,n-1 \\
                                                               d_0w_{HH}(r_0 \otimes \ldots \otimes r_n) & i = n
                                                  \end{cases}\right\} = \tilde{d}_iw_{HH}(r_0\otimes \ldots \otimes r_n).
\end{aligned}$$

Furthermore $w_{HH}$ commutes with degeneracies by the following equalities
$$
\begin{aligned}
w_{HH} s_i(r_0\otimes \ldots \otimes r_n) & = w_{HH}(r_0\otimes\ldots \otimes r_i\otimes 1 \otimes r_{i+1} \otimes \ldots\otimes r_n)\\
&= r_0\otimes r_n \otimes \ldots \otimes r_{i+1} \otimes 1\otimes r_i \otimes \ldots\otimes r_1\\
&= s_{n-i}(r_0\otimes r_n \otimes \ldots \otimes r_1) =\tilde{s}_iw_{HH}(r_0\otimes \ldots\otimes r_n).
 \end{aligned}
$$
The fact that $w_{HH}$ is a natural isomorphism is evident.
\end{proof}}

So far there is the following sequence of morphisms
$$\xymatrix{C_*(R) \ar[r]^{w_{HH}} & \tilde{C}_*(R^{op}) \ar[r]^{\tilde{C}_*(\tau)}  & \tilde{C}_*(R)}$$
This does not give an endomorphism of the Hochschild complex associated to $R$ (or its opposite complex), but the associated chain complexes give
a comparison.

\lemma{(cf. \cite[5.2.1]{LCy})
For $M_*$ a simplicial module and $\tilde{M}_*$ the associated opposite module (defined via $\tilde{d}_i := d_{n-i}$ and $\tilde{s}_i := s_{n-i}$ as before),
there is an isomorphism of the associated chain complexes $j\colon M_* \rightarrow \tilde{M}_*$ given by $j_n:=(-1)^{\frac{n(n+1)}{2}}\mathit{id}$.

\begin{proof}
The following equalities hold
$$
\begin{aligned}
(-1)^{\frac{n(n+1)}{2}}\tilde{d}&=(-1)^{\frac{n(n+1)}{2}}\sum_{i=0}^n (-1)^i \tilde{d}_i \\&=(-1)^{\frac{n(n+1)}{2}} \sum_{i=0}^n (-1)^i d_{n-i} \\
 &=(-1)^{\frac{n(n+1)}{2}} \sum_{i=0}^n (-1)^{n-i} d_i \\&= (-1)^{\frac{n(n+1)}{2}}(-1)^n \sum_{i=0}^n(-1)^i d_i \\
&= (-1)^{\frac{n(n+3)}{2}}d=(-1)^{\frac{(n-1)n}{2}}d
\end{aligned}$$
which give the result that $(-1)^{\frac{n(n+1)}{2}}$ is indeed a chain map.
\end{proof}\label{Opp=Vz}}

\rem{This lemma informally gives that opposing the simplicial structure changes the associated chain complex only up to sign.
This implies the following result.}

\thm{(cf. \cite[E.5.2.2]{LCy}) The Hochschild homology groups of any ring $R$ and its opposed ring $R^{op}$ are naturally isomorphic
 $$HH_*(R) \cong HH_*(R^{op}).$$
\begin{proof}
By lemma \ref{HH(op)} and lemma \ref{Opp=Vz} the isomorphism is given by composition of the simplicial isomorphism $w_{HH}$ and the chain isomorphism $j$ as follows
$$\xymatrix{C_*(R) \ar[r]^{w_{HH}} & \tilde{C}_*(R^{op}) \ar[r]^{j}&C_*(R^{op}) },$$
which gives the result.
\end{proof}}

Much more importantly the identification \ref{Opp=Vz} allows to induce an involution on Hochschild homology for rings with involution.

\cor{Let $R$ be a ring with involution $\tau$. Then there is an induced map on its Hochschild homology, which is given by the following composition
$$\xymatrix{C_*(R) \ar[r]^{w_{HH}} & \tilde{C}_*(R^{op}) \ar[r]^{\tilde{C}_*(\tau)}  & \tilde{C}_*(R) \ar[r]^j & C_*(R).}$$
Call this map the involution induced by $\tau$ on Hochschild homology.
\label{inducedHH}}

Beware that this is not a simplicial statement, but one in chain complexes. Furthermore the
choice of sign is not the only possible choice, but in this case dictated by the application to the trace map.\\[1ex]

In \ref{degreeiso} I focused on extensions of the integers by a prime root of unity and proved that the Dennis trace
map is non-trivial in degree $1$. Investigate the two evident involutions on their Hochschild homology in degree $1$.

\thm{For $\Z\lbrack \zeta_p \rbrack \cong \Z\lbrack X \rbrack/(\varphi_p)$ an extension of the integers by a $p$-th root of unity
for $p\in \N$ a prime number, there are the following involutions on Hochschild homology:
\begin{enumerate}
 \item For $\overline{~\cdot~}$ the map induced by complex conjugation, i.e.
$$
\begin{aligned}
 \overline{~\cdot~}\colon &\Z\lbrack X\rbrack /(\varphi_p) \rightarrow \Z\lbrack X\rbrack /(\varphi_p)\\
&\left[\sum\limits_{i=0}^{p-1}a_iX^i\right]\mapsto \left[\sum\limits_{i=0}^{p-1}a_iX^{p-i}\right],
\end{aligned}
$$
the induced involution on $HH_1(\Z\lbrack X\rbrack /(\varphi_p))$ is trivial.
\item The identity map induces a non-trivial involution on the first Hochschild homology group.
\end{enumerate}

\begin{proof}
1. On generators of the form $\lbrack \lbrack X^i \rbrack \otimes \lbrack X^j \rbrack \rbrack$ complex conjugation induces a map as follows
$$(\overline{~\cdot~})_*(\lbrack \lbrack X^i \rbrack \otimes \lbrack X^j \rbrack \rbrack)=(-1)^{\frac{1(1+1)}{2}}(\lbrack \lbrack X^{p-i} \rbrack \otimes \lbrack X^{p-j} \rbrack \rbrack),$$
which by the isomorphism of theorem \ref{degreeiso} reduces as follows
$$\pi_*((\overline{~\cdot~})_*(\lbrack \lbrack X^i \rbrack \otimes \lbrack X^j \rbrack \rbrack))=-\lbrack p-j\rbrack = \lbrack j \rbrack =
\pi_*(\lbrack \lbrack X^i \rbrack \otimes \lbrack X^j \rbrack \rbrack).$$
Since $\pi_*$ is an isomorphism by theorem \ref{degreeiso} this implies that $(\overline{~\cdot~})_*$ is the identity.

2. For the identity map calculate the following
$$(id)_*(\lbrack \lbrack X^i \rbrack \otimes \lbrack X^j \rbrack \rbrack)=-\lbrack \lbrack X^{i} \rbrack \otimes \lbrack X^j \rbrack \rbrack,$$
and $\pi_*(-\lbrack \lbrack X^{i} \rbrack \otimes \lbrack X^j \rbrack \rbrack)=-\lbrack j \rbrack = \lbrack p-j \rbrack$ implies that the
identity induces the inverse map on $\Z/p\Z$.
\end{proof}}

\section{The Dennis Trace Map Commutes with Involutions}
I first present how the induced involution on Hochschild homology of a ring with involution can be unravelled. In particular recall the
identification $(R\lbrack G \rbrack)^{op} = R\lbrack G^{op}\rbrack$ for a commutative coefficient ring $R$ and any
group $G$ (cf. Proposition \ref{canInv}). Furthermore recall the opposition of structures given by inverting, transposition
and pointwise involution (cf. Lemma \ref{tools}) and the opposing of simplicial structure as in definition \ref{Hochopp}.
Note that $S_*$ does not denote singular chains but cellular chains, $B_*$ is the bar complex introduced
in chapter 1 and $C_*$ is the Hochschild complex defined in section 1 of this chapter.
Finally $~{\tilde{\cdot}}~$ denotes the respective opposite simplicial structure for each of these.
Inspect the following diagram

\begin{landscape}
$$
\xymatrix{
S_*(|BGL_r(R)|,\Z) \ar[r]^{\tilde{}} \ar[d]^{S_*|w_{Bar}|}& B_*(\Z\lbrack GL_r(R) \rbrack) \ar[rr]^{\mathrm{inc}} \ar@{}[drr]|{(1)} \ar[d]^{w_{Bar}} & &
                C_*(\Z\lbrack GL_r(R) \rbrack) \ar[r]^{\mathrm{fus}} \ar[d]^{w_{HH}}  & C_*(M_r(R)) \ar@{}[ddr]^{(2)} \ar[d]^{w_{HH}} \ar[r]^{tr} &  C_*(R) \ar[dd]^{w_{HH}} \\
S_*(|\tilde{B}GL_r(R)^{op}|,\Z) \ar[r]^{\tilde{}} \ar[d]^{S_*|\tilde{B}(T)|}& \tilde{B}_*(\Z\lbrack GL_r(R)^{op}\rbrack ) \ar[r]^{\mathrm{inc}} \ar[d]^{\tilde{B}_*(\Z\lbrack T \rbrack)} &\tilde{C}_*(\Z\lbrack GL_r(R)^{op} \rbrack) \ar@{=}[r] \ar[d]^{\tilde{C}_*(\Z\lbrack T \rbrack)} &
                \tilde{C}_*(\Z\lbrack GL_r(R) \rbrack^{op})  \ar[r]^{\mathrm{fus}}& \tilde{C}_*(M_r(R)^{op})\ar[d]^{\tilde{C}_*(T)}&\\
S_*(|\tilde{B}GL_r(R^{op})|,\Z) \ar[r]^{\tilde{}} \ar[d]^{S_*|\tilde{B}(GL_r(\tau))|} & \tilde{B}_*(\Z\lbrack GL_r(R^{op})\rbrack ) \ar[r]^{\mathrm{inc}} \ar[d]^{\tilde{B}_*(\Z\lbrack GL_r(\tau) \rbrack)} &\tilde{C}_*(\Z\lbrack GL_r(R^{op}) \rbrack) \ar[d]^{\tilde{C}_*(\Z\lbrack GL_r(\tau) \rbrack)} \ar[rr]^{\mathrm{fus}}
                &              &\tilde{C}_*(M_r(R^{op}))\ar[d]^{\tilde{C}_*(M_r(\tau))} \ar[r]^{tr}& \tilde{C}_*(R^{op}) \ar[d]^{\tilde{C}_*(\tau)}\\
S_*(|\tilde{B}GL_r(R)|,\Z) \ar@{}[dr]|{(3)} \ar[r]^{\tilde{}} \ar[d]^{S_*(\Gamma)} & \tilde{B}_*(\Z\lbrack GL_r(R)\rbrack ) \ar[r]^{\mathrm{inc}} \ar[d]^j&\tilde{C}_*(\Z\lbrack GL_r(R) \rbrack) \ar[rr]^{\mathrm{fus}} \ar[d]^j
                &              &\tilde{C}_*(M_r(R)) \ar[r]^{tr} \ar[d]^j& \tilde{C}_*(R) \ar[d]^j \\
S_*(|BGL_r(R)|,\Z) \ar[r]^{\tilde{}} & B_*(\Z\lbrack GL_r(R)\rbrack ) \ar[r]^{\mathrm{inc}} &C_*(\Z\lbrack GL_r(R) \rbrack) \ar[rr]^{\mathrm{fus}}
                &              &C_*(M_r(R)) \ar[r]^{tr}& C_*(R)
}\label{diagr}
$$
\end{landscape}

\lemma{The diagram above commutes.

\begin{proof}
Most squares are instances of naturalities. More precisely the naturality of the isomorphism of theorem \ref{alleshomologie} in the left-most column,
and naturality of the inclusion map $\mathrm{inc}$, the fusion map $\mathrm{fus}$ and the trace map $\mathrm{tr}$ . There are only three exceptions, namely:

The complete reversal of coordinates in the bar complex is included into fixing the first
coordinate and reversing the remaining coordinates (top middle). Furthermore the generalised trace map
does not notice transposition of matrices (top right). Finally the homeomorphism of the classifying space
to its opposite is coherent with the isomorphism $j$ of chain complexes (bottom left).

(1) Let $\circ$ denote the multiplication in the opposite group, this yields
$$\begin{aligned}
(\mathrm{inc}\circ w_{Bar}) (g_1\otimes \ldots \otimes g_n) &= \mathrm{inc} (g_n \otimes \ldots \otimes g_1) \\&= (g_n \ldots g_1)^{-1}\otimes g_n \otimes \ldots \otimes g_1\\
&= (g_1 \circ \ldots \circ g_n)^{-1} \otimes g_n \otimes \ldots \otimes g_1 \\&= w_{HH}((g_1\circ\ldots \circ g_n)^{-1} \otimes g_1 \otimes \ldots \otimes g_n) \\&= (w_{HH}\circ \mathrm{inc})(g_1\otimes \ldots \otimes g_n).
\end{aligned}$$

(2) The familiar invariance of the trace map under transposition extends to this context as follows
$$ \begin{aligned}
(tr\circ \tilde{C}_*(T)\circ w_{HH})(A^0\otimes \ldots \otimes A^n) &= (tr \circ \tilde{C}_*(T))(A^0 \otimes A^n\otimes \ldots \otimes A^1)\\
&=  tr(T(A^0)\otimes T(A^n) \otimes \ldots \otimes T(A^1))\\
&=\sum_{i\in J} T(A^0)_{i_0,i_1}\otimes T(A^n)_{i_1,i_2}\otimes \ldots \otimes T(A^1)_{i_n,i_0} \\
&= \sum_{i \in J} {A^0}_{i_1,i_0} \otimes {A^n}_{i_2,i_1} \otimes \ldots \otimes {A^1}_{i_0,i_n}\\
&=w_{HH}\left(\sum_{i\in J} {A^0}_{i_1,i_0}\otimes {A^1}_{i_0,i_n}\otimes \ldots \otimes {A^n}_{i_2,i_1}\right)\\
&=(w_{HH}\circ tr)(A^0 \otimes \ldots \otimes A^n),
\end{aligned}$$
so the transposition just permuted the summands, which does not have any effect on the sum.

(3) In order to check, whether the square (3) commutes, inspect what the map
$$\begin{aligned}\Gamma \colon & BGL_r(R) \rightarrow \tilde{B}GL_r(R)\\
   &\lbrack x,(t_0,\ldots,t_n) \rbrack \mapsto \lbrack x,(t_n,\ldots,t_0)\rbrack
  \end{aligned}$$
does to orientations of cells, since in this context there is a meaningful way of speaking of equal bases in the cellular complexes of
$BGL_r(R)$ and $\tilde{B}GL_r(R)$.
The induced map yields
$$S_*(\Gamma)(e_x) = (-1)^{\frac{n(n+1)}{2}}e_x$$
for each cell in $BGL_r(R)$, since this amounts to calculating the degree of the map
$$\begin{aligned}
   &\Delta^n/(\partial \Delta^n) \rightarrow \Delta^n/(\partial \Delta^n)\\
   &\lbrack t_0,\ldots,t_n\rbrack \mapsto \lbrack t_n,\ldots, t_0\rbrack
  \end{aligned}$$
and that is precisely $(-1)^{\frac{n(n+1)}{2}}$ as chosen before for $j$ in the following columns. Hence it follows that the whole diagram commutes.
\end{proof}}

This process stabilises as well, so the result extends to $GL(R)$.

\rem{Recall from lemma \ref{homotopyinvrev} that reversal of coordinates $\kappa$ and the inverse map $|B\iota|$
are naturally homotopic as maps from $|BG|$ to $|B(G^{op})|$.

Notice furthermore that $\kappa$ is equal to the composition of reversing the simplex coordinates $\Gamma$ and reversal of the
coordinates in the bar complex $w_{Bar}$ as follows
$\kappa=\Gamma \circ |w_{Bar}| = |w_{Bar}|\circ\Gamma$, which commute, since each map reverses just one type of coordinate.
Furthermore $\Gamma$ commutes with each map which is induced by a group homomorphism, in other words, $\Gamma$ is a natural
transformation $\Gamma\colon |B(\_)| \Rightarrow |\tilde{B}(\_)|$.}

These results provide almost every coherence needed in order to show that the Dennis trace map commutes with the induced involutions defined before.

\thm{\label{derKernueberhaupt}The Dennis trace map is a natural transformation from the $K$-theory of rings with anti-involution to the
Hochschild homology of rings with anti-involution. In particular $Dtr$ transfers the involution given on $K$-theory
to the one given on Hochschild homology.

\begin{proof}
By the first of the preceding lemmas the diagram above (\ref{diagr}) reduces to the following square
$$
\xymatrix{
S_*(|BGL_r(R)|,\Z) \ar[r]^-{\overline{Dtr}} \ar[d]^{J}  &  C_*(R) \ar[d]^{\tau_*} \\
S_*(|BGL_r(R)|,\Z) \ar[r]^-{\overline{Dtr}}  &  C_*(R)
}
$$
with $J:=S_*(\Gamma)\circ S_*(|\tilde{B}(GL_r(\tau))|)\circ S_*(|\tilde{B}(T)|) \circ S_*(|w_{Bar}|)$ the left most sequence of maps and
$\tau_*\colon C_*(R) \rightarrow C_*(R)$ being the induced involution on the Hochschild complex (cf. Definition \ref{inducedHH}). Thus
the claim reduces to showing that $J$ indeed induces the same map as the involution on $GL_r(R)$.

Since $S_*$ is a functor, find $$J = S_*(\Gamma \circ |\tilde{B}(GL_r(\tau))| \circ |\tilde{B}(T)| \circ |w_{Bar}|)$$
but $|\cdot |$ is a functor as well, so $$J = S_*(\Gamma \circ |\tilde{B}(GL_r(\tau))\circ\tilde{B}(T)\circ w_{Bar}|)$$
$\tilde{B}$ is a functor, hence $$J = S_*(\Gamma \circ |\tilde{B}(GL_r(\tau)\circ T) \circ w_{Bar}|).$$
But $\Gamma$ is a natural transformation between the geometrical realisations of the bar construction and its opposite, which implies
$$J=S_*(|B(GL_r(\tau)\circ T)|\circ |w_{Bar}| \circ \Gamma)$$ and by the earlier remark this is
$$J=S_*(|B(GL_r(\tau)\circ T)|\circ \kappa).$$
Applying homology to the chain complexes in the diagram introduces the liberty to take another representative
for that map. Lemma \ref{homotopyinvrev} gives $\kappa \simeq |B\iota|$ and this yields the equations
$H_*(J) = H_*(|B(GL_r(\tau)\circ T)|\circ \kappa) = H_*(|B(GL_r(\tau)\circ T)|\circ |B\iota|)=H_*(|B(GL_r(\tau)\circ T \circ \iota)|).$
But the last term is the induced involution on $BGL_r(R)$. Stabilised with respect to $r$ this gives the following diagram
$$\xymatrix{
K_*(R)\ar[r]^{h} \ar[d]^{\tau_*}& H_*(GL(R)) \ar[r]^-{\overline{Dtr}} \ar[d]^{\tau_*} & HH_*(R) \ar[d]^{\tau_*}\\
K_*(R)\ar[r]^{h} & H_*(GL(R)) \ar[r]^-{\overline{Dtr}}& HH_*(R).\\
}$$
This proves the claim, because the first square commutes by naturality of the Hurewicz map and the second by the previous calculation.
Therefore the Dennis trace map is a natural transformation of functors from unital rings with anti-involution to the category of abelian groups.
\end{proof}}

\rem{In other words I have now proved that the Dennis trace map can be used to detect non-trivial involutions on $K$-theory in arbitrary degrees
by calculating induced involutions on Hochschild homology, which is easier in general. Nonetheless one still has to show that the trace map is non-trivial
and that the involution is non-trivial on the image of the trace map. Both restrictions can hinder the detection of a non-trivial involution.}

\section{Detecting Involutions with the Trace Map}
I have already shown in \ref{nontrivroot} for $\Z\lbrack \zeta_p\rbrack$ that the Dennis trace map is non-trivial. In particular the
involution on $K$-theory given by the identity on $\Z\lbrack \zeta_p \rbrack$ cannot be trivial, because the involution on the first
Hochschild homology group is non-trivial and the Dennis trace map commutes with the induced involutions. This is the case one would
hope for, since the non-triviality of the involution in $K$-theory can be detected by calculating the involution in Hochschild homology.

I want to emphasise that the trace map drastically simplified this detection. If one instead wanted to calculate the
induced involution on $K_1(\Z\lbrack \zeta_p\rbrack)\cong (\Z\lbrack \zeta_p\rbrack)^\times \oplus SK_1(\Z\lbrack \zeta_p\rbrack),$
this involves calculating the units, which is a famous classical result, the Dirichlet Unit Theorem
(cf. Rosenberg \cite{rosenberg1994algebraic} Theorem 2.3.8), and it needs $SK_1(\Z\lbrack \zeta_p\rbrack)=0$,
 which according to Rosenberg again ``is not an easy theorem and there doesn't seem to be an elementary proof'' (\cite{rosenberg1994algebraic} Remarks
after Theorem 2.3.8.\\[2ex]

However the Dennis trace map does not always detect non-trivial involutions. In order to investigate the involution on Hochschild homology of
Laurent polynomials study the first homology group.

\rem{This part is adapted from Prop 1.1.10 \cite{LCy}. For a commutative ring $R$ there is the following interpretation of its first
Hochschild homology. Everything in degree $1$ is a cycle, if $R$ is commutative. The boundaries introduce the following relation
$$\begin{aligned}
R \otimes R \otimes R &\rightarrow R \otimes R,\\
p \otimes q \otimes r &\mapsto pq\otimes r - p\otimes qr + rp \otimes q.
  \end{aligned} $$
This inspires the interpretation of $HH_1(R)$ as K\"ahler-Differentials $\Omega^1(R)$ for a commutative ring
(cf. \cite{LCy} 1.1.9/10), so write $a d b := \lbrack a\otimes b\rbrack$ and $da = 1da$, which implies the relation
$$a d(bc) = ab d(c) + ac d(b)$$
such that the relation introduced by the boundaries is a Leibniz-rule as in differentials.

In particular this gives the usual result $dr^n = nr^{n-1}dr$ for each $r\in R$ and $n \in \N$, and for units in $R$ even for each $n\in \Z$.
Furthermore for $r\in R^\times$ the equality $0 = d1 = d(rr^{-1})= rdr^{-1} + r^{-1}dr$ gives $rdr^{-1} = -r^{-1}dr$.}

By \ref{invR1} the involution induced by $t\mapsto t^{-1}$ on $K_1(R\lbrack t^\pm \rbrack)$ is
non-trivial. The following example investigates, whether the Dennis trace map detects that.

\ex{By the relation $dr^n = nr^{n-1}dr$ it is clear that $HH_1(R\lbrack t^\pm\rbrack)$ is generated by the elements $(dt,(dr)_{r\in R})$. The first
$K$-groups has the trivial units as a subgroup $K_1(R\lbrack t,t^{-1}\rbrack) \supset R^\times \oplus \Z$.

For euclidean rings $R$ it is actually even true that $K_1(R\lbrack t^\pm\rbrack) \cong R^\times \oplus \Z$, because there is a
general splitting result for $K$-theory of Laurent rings, if the category of $R$-modules is well-behaved (cf. \cite{srinivas1996algebraic} Theorem 5.2).

For a trivial unit $rt^k$ as an element of $K_1(R\lbrack t^\pm\rbrack)$ the preceding example gives the following equations
$$\begin{aligned}
Dtr(rt^k) &= tr(\mathrm{fus}(\mathrm{inc}(rt^k)))\\
 &= tr(\mathrm{fus}(r^{-1}t^{-k}drt^k))\\
 &= tr(\mathrm{fus}(r^{-1}t^{-k}(rdt^k+t^kdr)))\\
 &= tr(\mathrm{fus}(t^{-k}dt^k+r^{-1}dr))\\
 &= tr(\mathrm{fus}(kt^{-k}t^{k-1}dt+r^{-1}dr))\\
 &= tr(\mathrm{fus}(kt^{-1}dt+r^{-1}dr))=kt^{-1}dt+r^{-1}dr.\\
\end{aligned}$$
Furthermore on elements of this form the involution yields
$$\tau_*(kt^{-1}dt+r^{-1}dr) = -(ktdt^{-1}+r^{-1}dr)= k(-tdt^{-1})-r^{-1}dr =kt^{-1}dt - r^{-1}dr.$$
So the involution on Hochschild homology of $R\lbrack t^\pm\rbrack$ is non-trivial, if there is a unit in $R$ with $r^{-1}dr\neq0$.
In particular this gives a specific example, which provides an instance of a non-trivial involution on $K$-theory, which is not detected
by the trace map, namely for $R=\Z$. In $\Z$ the only units are $\{\pm 1\}$ which give $1d(-1)=(-1)d1 = 0$. The
result mentioned above even shows $K_1(\Z\lbrack t^\pm\rbrack) \cong \Z\lbrack t^\pm \rbrack^\times.$
Thus there are no further elements in $K_1$ and so I described the involution on the complete image of the trace map $\mathrm{im}(\mathrm{Dtr})$. But here the involution is trivial,
even though it is not trivial on $K_1(\Z\lbrack t^\pm\rbrack)$.}

I will close these examples with the strongest defect the trace map can have.

\ex{By theorem \ref{HH=h} Hochschild homology of the group ring $\Z\lbrack G\rbrack$ is isomorphic to group homology
$$HH_*(\Z\lbrack G\rbrack)=H_*(G).$$
This implies that in principle an involution on $K$-theory of $\Z\lbrack G\rbrack$ can be detected by computing group homology, which in most cases is
a feasible task. But for $\Z/p\Z$, there is the following well-known result (cf. Weibel \cite{weibel1995introduction} Example 6.2.3)
$$H_n(\Z/p\Z)\cong\begin{cases}
                   \Z & \mathit{~~~for~~} n = 0\\
		    \Z/p\Z & \mathit{~~~for~~} n \mathit{~~odd}\\
		    0 & \mathit{~~~for~~} n\neq 0 \mathit{~~even},
                  \end{cases}
$$
which in this case yields that the trace map is trivial for each even degree not equal to zero.
Thus in even degrees the trace map cannot detect, whether an induced involution on $K$-theory of the group ring $K_{2n}(\Z\lbrack G\rbrack)$
is non-trivial.}

\rem{As a final remark I want to summarise the process of showing non-triviality of induced involutions on $K$-groups from the point of view
of this diploma thesis: One would follow the strategy to consider a strict bimonoidal category, which might be a combinatorial model for
$K$-theory of the object of interest (for example $ko$ or $ku$ as in \cite{richter2010involution}), then project to its components and group complete
additively. This is a ring and hence allows to study the induced involution on the Hochschild homology of this ring. If this involution is non-trivial, one can try to
retrace such a class back to the $K$-theory of the given bimonoidal category, which is an element not fixed by the induced involution on the $K$-groups of the bimonoidal category.}
